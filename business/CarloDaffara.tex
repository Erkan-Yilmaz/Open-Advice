\chapter{Free and Open Source based business models - Carlo Daffara}

\textit{Carlo Daffara is a researcher in the field of open source-based business
models, collaborative development of digital artifacts, and open source software
employment in companies. He is part of the editorial review board of the
International Journal of Open Source Software \& Processes (IJOSSP) and member
of the technical board of two regional Open Source competence centers, as well
as member of the FSFE European Legal Network. He has been part of SC34 and JTC1
committees in the Italian branch of ISO, UNINFO; and participated in the
Internet Society Public Software working group, and many other
standardization-related initiatives. Previous to that, Carlo Daffara was the
Italian representative at the European Working Group on Libre Software, the
first EU initiative in support of Open Source and Free Software. He chaired the
SME working group of the EU Task Force on Competitiveness, and the IEEE open
source middleware working group of the Technical Committee on Scalable Computing. He worked
as project reviewer for the EC in the field of international collaboration,
software engineering, open source and distributed systems and was Principal
Investigator in several EU research projects.}

\section*{Introduction}

``How do you make money with Free Software?'' was a very common question just a
few years ago. Today, that question has evolved into ``What are successful
business strategies that can be implemented on top of Free Software?'' The
question is not as gratuitous as it may seem, as many academic researchers still
write this kind of text: ``Open-source software is deliberately developed outside
of market mechanisms \dots fails to contribute to the creation of value in
development, as opposed to the commercial software market \dots does not generate
profit, income, jobs or taxes \dots The open-source licenses on the software aim to
suppress any ownership claims to the software and prevent prices from being
established for it. In the end, the developed software cannot be used to
generate profit.'' [Koot 03] or [Eng 10] that claims that ``economists showed that
real world open source collaborations rely on many different incentives such as
education, signaling, and reputation.'' (without any mention of economic
incentives). Is it possible to do better? Is there a reality behind the idea of
Free/Open Source businesses?

\section*{Open Source and Economic Realities}

In most areas, the use of Free Software brings a substantial economic advantage,
thanks to the shared development and maintenance costs, already described by
researchers like Gosh, that estimated an average R\&D cost reduction of 36\%.
The large share of ``internal'' Free Software deployments explains why some of the
economic benefits are not perceived directly in the business service market, as
shown by Gartner:\todo{image}

In a similar way, the FLOSSIMPACT study found in 2006 that firms contributing
code to FLOSS projects have in total at least 570 thousand employees and annual
revenue of 263B Euro [Gosh 06], thus making Open Source and Free Software among the
most important ICT-based economic phenomenons. It is important also to recognize
that a substantial percentage of this economic value is not immediately visible
in the marketplace, as the majority of software is not developed with the intent
of selling it (the so-called ``shrinkwrap'' software) but is developed for
internal use only. As identified by the FISTERA EU thematic network in fact the
majority of software is developed for internal use only:\todo{table}

It is clear that what we call ``the software market'' is in reality much smaller
than the real market for software and services, and that 80\% of it is
invisible. We will see that Open Source has a major part of the economic market
directly through this ``internal'' development model.

\section*{Business Models and Value Proposition}

The basic idea behind business models is quite simple: I have something or can
do something -- the ``value proposition'' -- and it is more economical to pay me to
do or get this ``something'' instead of doing it yourself (sometimes it may even
be impossible to find alternatives, as in natural or man-made monopolies, so the
idea of doing it myself may not be applicable)
There are two possible sources for the value: a property (something that can be
transferred) and efficiency (something that is inherent in what the company does,
and how they do it). With Open Source, usually ``property'' is non-exclusive (with
the exception of what is called ``Open Core'', where part of the code is not open
at all, and that will be covered later in the article). Other examples of
property are trademarks, patents, licenses \dots anything that may be
transferred to another entity through a contract or legal transaction.
Efficiency is the ability to perform an action with a lower cost (both tangible
and intangible), and is something that follows the specialization in a work area
or appears thanks to a new technology. Examples of the first are simply the
decrease in time necessary to perform an action when you increase your expertise
in it; the first time you install a complex system it may require a lot of effort,
and this effort is reduced the more experience you have with the tasks necessary to
perform the installation itself; examples of the second may be the introduction
of a tool that simplifies the process (for example, through image cloning) and
it introduces a huge discontinuity, a ``jump'' in the graph of efficiency versus
time.
These two aspects are the basis of all the business models that we have analyzed
in the past; it is possible to show that all of them fall in a continuum between
properties and efficiency:\todo{image}

Among the results of our past research project, one thing that we found is that
property-based projects tend to have lower contributions from the outside,
because it requires a legal transaction to become part of the company’s
properties; think for example about dual licensing: to become part of the product
source code, an external contributor needs to sign off his rights to the code,
to allow the company to sell the enterprise version alongside the open one.
On the other hand, right-handed models based purely on efficiency tends to have
higher contributions and visibility, but lower monetization rates. As I wrote
many times, there is no ideal business model, but a spectrum of possible models,
and companies should adapt themselves to changing market conditions and adapt
their model as well. Some companies start as pure efficiency based, and build an
internal property with time; some others may start as property based, and move
to the other side to increase contributions and reduce the engineering effort
(or enlarge the user base, to create alternative ways of monetizing users).

\section*{A Business Models Taxonomy}

The EU FLOSSMETRICS study on Free Software-based business model created, after
an analysis of more than 200 companies, a taxonomy of the main business models
used by Open Source companies; the main models identified in the market are: 
\begin{itemize}
 \item Dual licensing: the same software code distributed under the GPL and a
proprietary license. This model is mainly used by producers of
developer-oriented tools and software, and works thanks to the strong coupling
clause of the GPL, that requires derivative works or software directly linked to
be covered under the same license. Companies not willing to release their own
software under the GPL can obtain a proprietary license that provides an
exemption from the distribution conditions of the GPL, which seems desirable to
some parties. The downside of dual licensing is that external contributors must
accept the same licensing regime, and this has been shown to reduce the volume
of external contributions, which are limited mainly to bug fixes and small
additions.
 \item Open Core (previously called ``split Free Software/proprietary'' or
``proprietary value-add''): this model distinguishes between a basic Free Software
and a proprietary version, based on the Free Software one but with the addition
of proprietary plug-ins. Most companies following such a model adopt the Mozilla
Public License, as it allows explicitly this form of intermixing, and allows for
much greater participation from external contributions without the same
requirements for copyright consolidation as in dual licensing. The model has the
intrinsic downside that the Free Software product must be valuable to be
attractive for the users, i.e. it should not be reduced to ``crippleware'', yet at
the same time should not cannibalize the proprietary product. This balance is
difficult to achieve and maintain over time; also, if the software is of large
interest, developers may try to complete the missing functionality in Free
Software, thus reducing the attractiveness of the proprietary version and
potentially giving rise to a full Free Software competitor that will not be
limited in the same way.
 \item Product specialists: companies that created, or maintain a specific
software project, and use a Free Software license to distribute it. The main
revenues are provided from services like training and consulting (the“ITSC” ``ITSC''
class) and follow the original ``best code here'' and ``best knowledge here'' of the
original EUWG classification [DB 00]. It leverages the assumption, commonly
held, that the most knowledgeable experts on a software are those that have
developed it, and this way can provide services with a limited marketing effort,
by leveraging the free redistribution of the code. The downside of the model is
that there is a limited barrier of entry for potential competitors, as the only
investment that is needed is in the acquisition of specific skills and expertise
on the software itself.
 \item Platform providers: companies that provide selection, support,
integration and services on a set of projects, collectively forming a tested and
verified platform. In this sense, even GNU/Linux distributions were classified
as platforms; the interesting observation is that those distributions are
licensed for a significant part under Free Software licenses to maximize
external contributions, and leverage copyright protection to prevent outright
copying but not ``cloning'' (the removal of copyrighted material like logos and
trademark to create a new product)\footnote{Examples of RedHat clones are CentOS
and Oracle Linux.}. The main value proposition comes in the
form of guaranteed quality, stability and reliability, and the certainty of
support for business critical applications.
 \item Selection/consulting companies: companies in this class are not strictly
developers, but provide consulting and selection/evaluation services on a wide
range of project, in a way that is close to the analyst role. These companies
tend to have very limited impact on the Free Software communities, as the
evaluation results and the evaluation process are usually a proprietary asset.
 \item Aggregate support providers: companies that provide a one-stop support on
several separate Free Software products, usually by directly employing
developers or forwarding support requests to second-stage product specialists.
 \item Legal certification and consulting: these companies do not provide any
specific code activity, but provide support in checking license compliance,
sometimes also providing coverage and insurance for legal attacks; some
companies employ tools for verify that code is not improperly reused across
company boundaries or in an improper way.
 \item Training and documentation: companies that offer courses, on-line and
physical training, additional documentation or manuals. This is usually offered
as part of a support contract, but recently several large scale training center
networks started offering Free Software-specific courses.
 \item R\&D cost sharing: A company or organization may need a new or improved
version of a software package, and fund some consultant or software manufacturer
to do the work. Later on, the resulting software is redistributed as Open Source
to take advantage of the large pool of skilled developers who can debug and
improve it. A good example is the Maemo platform, used by Nokia in its Mobile
Internet Devices (like the N810); within Maemo, only 7.5\% of the code is
proprietary, with a reduction in costs estimated in 228M\$ (and a reduction in
time-to-market of one year). Another example is the Eclipse ecosystem, an
integrated development environment (IDE) originally released as Free Software by
IBM and later managed by the Eclipse Foundation. Many companies adopted Eclipse
as a basis for their own product, and this way reduced the overall cost of
creating a software product that provides in some way developer-oriented
functionality. There is a large number of companies, universities and individual
that participate in the Eclipse ecosystem; as an example:\todo{image} As
recently measured, IBM contributes for around 46\% of the project, with
individuals accounting for 25\%, and a large number of companies like Oracle,
Borland, Actuate and many others with percentages that go from 1 to 7\%. This is
similar to the results obtained from analysis of the Linux kernel, and show that
when there is a healthy and large ecosystem the shared work reduces engineering
cost significantly; in [Gosh 06] it is estimated that it is possible to obtain
savings in terms of software research and development of 36\% through the use of
Free Software; this is, in itself, the largest actual ``market'' for Free
Software, as demonstrated by the fact that the majority of developers are using
at least some Free Software within their own code (56.2\%, as reported in [ED
05]). Another excellent example of ``coopetition'' among companies is the WebKit
project, the HTML rendering engine that is at the basis of the Google Chrome
browser, Apple Safari and is used in the majority of mobile devices. If we
examine the contributions to the project, we can see that after a few months
co-development resources quickly increase and can even surpass those of the
project maintainer: \todo{image} The graph shows the amount of contributions
(patches) to the WebKit codebase by committers; as in the beginning the project
was managed by Apple developers, it clearly shows that after little more than
one year contributions from outside become larger-thus reducing the maintenance
costs and the engineering effort, thanks to the division of work among
co-developers.
 \item Indirect revenues: A company may decide to fund Free Software projects if
those projects can create a significant revenue source for related products, not
directly connected with source code or software. One of the most common cases is
the writing of software needed to run hardware, for instance, operating system
drivers for specific hardware. In fact, many hardware manufacturers are already
distributing gratis software drivers. Some of them are already distributing some
of their drivers (specially those for the Linux kernel) as Free Software. The
loss-leader is a traditional commercial model, common also outside of the world
of software; in this model, effort is invested in a Free Software project to
create or extend another market under different conditions. For example,
hardware vendors invest in the development of software drivers for Free Software
operating systems (like GNU/Linux) to extend the market of the hardware itself.
Other ancillary models are for example those of the Mozilla foundation, which
obtains a non trivial amount of money from a search engine partnership with
Google (an estimated 72M\$ in 2006), while SourceForge/OSTG receives the
majority of revenues from ecommerce sales of the affiliate ThinkGeek site
\end{itemize}

Some companies have more than one principal model, and thus are counted twice;
in particular, most dual licensing companies are also selling support services,
and thus are marked as both. Also, product specialists are counted only when
there is a demonstrable participation of the company into the project as ``main
committer''; otherwise, the number of specialists would be much greater, as some
projects are the center of commercial support from many companies (a good
example is OpenBravo or Zope).

Another relevant consideration is the fact that platform providers, while
limited in number, tend to have a much larger revenue rate than both specialists
or open core companies. Many researchers are trying to identify whether there is
a more ``efficient'' model among all those surveyed; what we found is that the
most probable future outcome will be a continuous shift across model, with a
long-term consolidation of development consortia (like the Eclipse or Apache
consortium) that provide strong legal infrastructure and development advantages,
and product specialists that provide vertical offerings for specific markets. 

\section*{Conclusions}

Open Source not only allows for sustainable, and even very large market presence
(RedHat is already quite near to 1B\$ in annual revenues) but also many
different models that are totally impossible with proprietary software. The fact
that Free Software is a non-rival good also facilitates cooperation between
companies, both to increase the geographic base and to be able to engage large
scale contracts that may require multiple competencies, both geographical (same
product or service, different geographical areas); ``vertical'' (among products)
or ``horizontal'' (among activities). This facilitation of new ecosystems is one
of the reasons why Open Source and Free Software are now present in nearly all
the IT infrastructures in the world, increasing value and helping companies and
Public Administrations in reducing costs and collaborating together for better
software.

\todo{biography}
