\chapter{Kick, Push - Andre Klapper}

\textit{In real life, Andre works for Openismus as a member of MeeGo Error
management and as maemo.org bugmaster. During lunch break or while
sleeping he works on random things in GNOME (Bugsquad, Release team,
Translation coordination team, etc), or studies, or eats ice cream.}

At the very beginning I only had one question: How can I print a part of the
email which I received in Gnome's email client Evolution? I asked on the
corresponding mailing list.

I had switched to Linux exactly one year ago, out of frustration that I
could not make my modem work after reinstalling a proprietary operating system
that was popular around that time.

The answer to my question was ``not possible''. Cool kids would have checked out
the code, compiled it, hacked it to make it act as wanted, and submitted a patch
attached to a bug report by then. Well, as you can guess I was not a cool kid.
My programming skills are rather limited, so I stuck to a cumbersome printing
workaround for the time being. The answer I received on the mailing list also
mentioned that the feature was in planning, and that a bug report had been filed
on my behalf (without mentioning where, but I did not care about that - I was
happy that there were plans to fix my issue soon).

It may just have been my laziness to have stayed subscribed to the mailing list.
Some folks mentioned the bug tracker from time to time, often as a direct
response to feature requests, so I took a look at it eventually. But bug
trackers, especially Bugzilla, are strange tools with lots of complex options.
An area you normally prefer to avoid unless you are a masochist. They contain
many tickets describing bugs or feature requests by users and developers. It
looked as if those reports were partially also used for planning priorities.
(Calling this ``Project Management'' would be an euphemism - less than one
fourth of the issues that were planned to get fixed or implemented for a
specific release actually got fixed in the end.)

What I found beside an interesting look at the issues of the software and the
popularity of certain requests were inconsistencies and some noise, like lots of
duplicates or bug reports missing enough information to get processed properly.
I felt like cleaning up a bit by ``triaging'' the available bug reports. I do
not know what this tells you about my mindset though - add wrong buzzwords for
random characteristics here, like organized, persistent or knowledgeable. Also
nice irony considering that my father always used to complain about my messy
room.

So back in those dial-up modem times I usually collected my questions and went
online to enter IRC once a day in order to shoot my questions at Evolution's
bugmaster who was always welcoming, patient and willing to share his experience.
If there had been a triaging guide available at that time covering basic bug
management knowledge and explaining good practices and common pitfalls: I had
not heard about it.

The amount of open reports decreased by 20\% within a few months though that was
of course not just because of one person starting to triage some tickets.
Obviously there was some work waiting to get picked up by somebody -- like
decreasing the amount of open tickets for the developers so that they could
better focus, discussing and setting some priorities with them, and responding
to some users' comments that had remained unanswered at that time. Open Source
is always welcoming to contributions once you have found your hook to
participate.

Way later I realized that there is some documentation around to dive into. Luis
Villa - who might have been the first bugmaster ever -- wrote an essay called
``Why Everyone Needs A Bugmaster''\footnote{{
http://tieguy.org/talks-files/LCA-2005-paper-html/index.html}}, and most
Bugsquad teams in Open Source projects were publishing triaging guides in the
meantime that helped newbies get involved in the community. Many Open Source
developers started their great Open Source careers by triaging bugs and gained
initial experience in software project management.

Nowadays there are also tools which can save you a lot of time when it comes to
the repetitive grunt work part of triaging. On the server side GNOME's ``stock
answers'' extension provides common and frequently used comments to add to
tickets via one click, and on the client side you can run your own Greasemonkey
scripts or Matěj Cepl's Jetpack extension called
``bugzilla-triage-scripts''\footnote{{
https://fedorahosted.org/bugzilla-triage-scripts}}.

If you are an average or poor musician but still love music more than anything
else, you can stick around in the business as a journalist. Software development also has such niches apart from the default idea of writing code that can make you happy. You have to spend some time to find them but it is worth the efforts, experience and contacts, and with some luck and skills it might even earn you a living in your personal field of interest and keep you from ending up as a code monkey.
