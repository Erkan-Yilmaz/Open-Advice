\chapter{What I wish I would have known about testing when I started... - Jonathan Leto}
\todo{bio}
When I first got involved in Free and Open Source Software, I had no clue what tests were or why they were important. I had worked on some personal programming projects before, but the first time I actually worked on a project with others, i.e., got a commit bit, was Yacas, a computer algebra system similar to Mathematica (Yet Another Computer Algebra System).

At this stage in my journey, tests were an afterthought. My general meta-algorithm was to hack on code $\rightarrow$ see if it works $\rightarrow$ write a simple test to show it works (optional). If a test was challenging to write, it most likely never got written.

This is the first step in the path to Test Driven Enlightenment. You know tests are probably a good idea, but one hasn't seen the benefit of them clearly, so they are only written occasionally.

If I could open up a wormhole and tell my younger self one piece of wisdom about testing, it would be:
\begin{quote}
Some tests, in the long-run, are more important than the code they test.                                                                        \end{quote}

A few people right about now may be thinking that I put on my tinfoil testing hat when I sit down to code. How can tests be \emph{more} important than the code they test? Tests are the proof that your code \emph{actually} works, and they guide you to writing correct code as well as providing the flexibility to change code and know that features still work. The larger your codebase becomes, the more valuable your tests are, because they allow you to change one part of your code and still be sure that the rest of it works.

Another vital reason to write tests is because it indicates that something is explicitly desirable, rather than an unintended side-effect or oversight. If you have a specification, you can use tests to verify that you meet it, which is very valuable, and in some industries, necessary. A test is just like telling a story, where the story is how you think code should work.

Code either changes and evolves or bitrots\footnote{The term "bitrot" is coder slang for the almost universal fact that if a piece of code doesn't change but everything it relies on does, it ``rots'' and usually has very little chance of working without modifications being made to accommodate newer software and hardware.}.

Very often, you will write tests once, but then totally refactor your implementation or even rewrite it from scratch. Tests often outlive the code they originally tested, i.e., one set of tests can be used no matter how many times your code is refactored. They are actually the litmus test that allows you to throw away an old implementation and say ``this newer implementation has a much better design and passes our test suite.'' I have seen this happen many times in the Perl and Parrot communities, where you can often find me.

Tests allow you to change things quickly and know if something is broken. They are like jet packs for developers.

Carpenters have a bit of sage wisdom that goes like this:
\begin{quote}Measure twice, cut once.\end{quote} 

Coding is like cutting and tests are like measuring.

Test Driven Enlightenment saves an enormous amount of time, because instead of flailing around, fiddling with code, not having a direction, tests hone your focus.

Tests also are very good positive feedback. Every time you make a new test pass, you know that your code is better and it has one more feature or one less bug.

It is easy to think ``I want to add 50 features'' and spend all day fiddling with code, constantly switching between working on different things. Most of the time, very little will be accomplished. Test Driven Enlightenment guides one to focus on making one test pass at a time.

If you have a single failing test, you are a mission to make it pass. It focuses your brain on something very specific, which very often has better results than switching between tasks constantly.

Most information about being test-driven is very specific to a language or situation, but that doesn't need to be the case. Here is the how to approach adding a new feature or fixing a bug in any language:
\begin{enumerate}
 \item Write a test that fails, which you think will pass when the feature is implemented or bug is fixed. Advanced: As you write the test, run it occasionally, even if it is not done yet, and guess the actual error message that the test will give. The more tests you write and run, the easier this will become.
 \item Hack on the code.
 \item Run the test. If it passes, go to \#4, otherwise go to \#2.
 \item You are done! Do a happy dance :)
\end{enumerate}

This method works for any kind of test and any language. If there is only one thing about testing that you remember from this essay, let it be the steps above.

Here are some more general test-driven guidelines that will serve you well and apply in almost any situation:
\begin{enumerate}
 \item Understand the difference between what is being tested and what is being used as a tool to test something else.
 \item Fragile tests. You could write a test that makes sure an error message is exactly correct. But what happens when the error message changes? What happens when someone internationalizes your code to Catalan? What happens when someone runs your code on an OS you've never heard of? The more resilient your test is, the more valuable it will be.
\end{enumerate}

Think about these things when you write tests. You want them to be resilient, i.e., tests, for the most part, should only have to change when functionality changes. If you have to change your tests often, but functionality is not changing, you are doing something wrong.

\section*{Kinds of tests}

Many people start to get confused when people speak of ``integration tests'', ``unit tests'', ``acceptance tests'' and many other flavors of tests. One shouldn't worry too much about these terms. The more tests you write, the more nuances you will see and the differences between tests will become more apparent. Every one does not have the same definition for what these tests are, but the terms are still useful to describe kinds of tests.

\section*{Unit Tests vs. Integration Tests}

Unit tests and integration tests form a spectrum. Unit tests test small bits of code, and integration tests verify how more than one unit fit together. The test writer gets to decide what comprises a unit, but most often it is at the level of a function or method, although some languages call those things by different names.

To make this a little more concrete, we will give a basic analogy using functions. Imagine that f(x)\todo{make this a proper formula} and g(x)\todo{make this a proper formula} are two functions which represent two units of code. For concreteness, let's assume they represent two specific functions in your favorite Free/Open Source projects' codebase.

An integration test asserts something like function composition, i.e., f(g(a)) = b\todo{make this a proper formula}. An integration test is testing how multiple things integrate or work together, instead of how a single part works individually. If algebra isn't your thing, another way to look at it is unit tests only test one part of the machine at a time, but integration tests very many parts work in unison. A great example of an integration test is test driving a car. You aren't checking the air pressure, or measuring voltage of the spark plugs. You are making sure the vehicle works as a whole.

Most of the time it is good to have both. I often start with unit tests and add integration tests as needed, since you will weed out the most basic bugs first, then find more subtle bugs that are related to how pieces don't quite fit together, as opposed to the individual pieces not working. Many people write integration tests first and then delve into unit tests. Which you write first isn't nearly as important as writing both kinds.
