\authorbio{Ara Pulido is a testing engineer working for Canonical, first as part of
the Ubuntu QA team, and now as part of the Hardware Certification team. Although
she started her career as a developer, she quickly found out that what she
really liked was testing the software. She is very interested in new testing
techniques and tries to apply her knowledge to make Ubuntu better.}

\chapterwithauthor{Ara Pulido}{Given Enough Eyeballs, Not All Bugs are Shallow}

\section*{Dogfooding Is Not Enough} 

I have been involved with Free Software since my early days at university in
Granada. There, with some friends, we founded the local Linux User
Group and organized several activities to
promote Free Software. But, since I left university, and until I started working
at Canonical, my professional career had been in the proprietary software
industry, first as a developer and after that as a tester.

When working in a proprietary software project, testing resources are very
limited. A small testing team continues the work that developers started with
unit testing, using their expertise to find as many bugs as possible, to release
the product in good shape for end user consumption. In the Free Software world,
however, everything changes.

When I was hired at Canonical, apart from fulfilling the dream of having a paid
job in a Free Software project, I was amazed by the possibilities that testing a
Free Software project brought. The development of the product happens in the
open, and users can access the software in the early stages, test it and file
bugs as they encounter them. For a person passionate about testing, this is a new
world with lots of new possibilities. I wanted to make the most of it.

As many people do, I thought that dogfooding, or using the software that you are
aiming to release, was the most important testing activity that we could do in
Free Software. But, if ``given enough eyeballs all the bugs are shallow'', (one of
the key lessons of Raymond's ``The Cathedral \& The Bazaar''), and Ubuntu had
millions of users, why were very important bugs still slipping into the release?

First thing that I found when I started working at Canonical was that the
organized testing activities were very few or nonexistent. The only testing
activities that were somehow organized were in the form of emails sent to a
particular mailing list calling for testing a package in the development version
of Ubuntu. I do not believe that this can be considered a proper testing
activity, but just another form of dogfooding. This kind of testing generates a
lot of duplicated bugs, as a really easy to spot bug will be filed by hundreds
of people. Unfortunately, the really hard to spot but potentially critical bug,
if someone files it, is likely to remain unnoticed, due to the noise created by
the other hundreds of bugs.

\section*{Looking better}

Is this situation improving? Are we getting better at testing in Free Software projects?
Yes, I really believe so.

During the latest Ubuntu development cycles we have started several organized
testing activities. The range of topics for these activities is wide, including
areas like new desktop features, regression testing, X.org drivers testing or
laptop hardware testing. The results of these activities are always tracked, and
they prove to be really useful for developers, as they are able to know if the
new features are working correctly, instead of guessing that they work correctly
because of the absence of bugs.

Regarding tools that help testing, many improvements have been made:
\begin{itemize}
 \item Apport\footnote{\url{http://wiki.ubuntu.com/Apport}} has contributed to
increase the level of detail of the bugs reported against Ubuntu: crashers
include all the debugging information and their duplicates are found and marked
as such; people can report bugs based on symptoms, etc.
 \item Launchpad\footnote{\url{http://launchpad.net}}, with its upstream
connections, has allowed having a full view of the bugs, knowing that bugs
happening in Ubuntu are usually bugs in the upstream projects, and allowing
developers to know if the bugs are being solved there. 
 \item Firefox, with its Test Pilot extension and program, drives the testing
without having to leave the
browser\footnote{\url{http://testpilot.mozillalabs.com}}. This is, I believe, a
much better way to reach testers than a mailing list or an IRC channel.
 \item The Ubuntu QA team is testing the desktop in an automated fashion and
reporting results every
week\footnote{\url{http://reports.qa.ubuntu.com/reports/desktop-testing/natty}},
allowing developers to have a very quick way to check that there have not been
any major regressions during the development.
\end{itemize}

Although testing in Free Software projects is getting better, there is still a lot to be
done.

\section*{Looking ahead}

Testing is a skilled activity that requires lots of expertise, but in the Free Software
community is still seen as an activity that does not require much effort. One of
the reasons could be that the way we do testing is still very old-fashioned and
does not reflect the increase of complexity in the Free Software world in the
last decade. How can it be possible that with the amount of innovation that we
are generating in Free Software communities, testing is still done like it was in
the 80s? Let's face it, fixed testcases are boring and get easily outdated. How
are we going to grow a testing community, who is supposed to find meaningful
bugs if their main required activity is updating testcases?

But, how do we improve testing? Of course, we cannot completely get rid of
testcases, but we need to change the way we create and maintain them. Our
testers and users are intelligent, so, why creating step-by-step scripts? Those
could easily get replaced by an automated testing tool. Instead of that, let's
just have a list of activities you perform with the application and some
properties it should have, for example, ``Shortcuts in the launcher can be
rearranged'' or ``Starting up LibreOffice is fast''. Testers will figure out how
to do it, and will create their testcases as they test.

But this is not enough, we need better tools to help testers know what to test,
when and how.  What about having an API to allow developers to send messages to
testers about updates or new features that need testing? What about an
application that tell us what part of our system needs testing based on testing
activity? In the case of Ubuntu we have the data in Launchpad (we would need
testing data as well, but at least we have bug data). If I want to start a
testing session against a particular component I would love to have the areas
that have not been tested yet and a list of the five bugs with more duplicates
for that particular version, so I avoid filing those again. I would love to have
all this information without leaving the same desktop that I am testing. This is
something that Firefox has started with Test Pilot, although they are currently
mainly gathering browser activity.

Communication between downstream and upstream and vice-versa also needs to get
better. During the development of a distribution, many of the upstream versions
are also under development, and they already have a list of known bugs. If I am
a tester of Firefox through Ubuntu, I would love to have a list of known bugs as
soon as the new package gets to the archive. This could be done by having an
acknowledged syntax for release notes, that could then get easily parsed and
bugs automatically filed and connected to the upstream bugs. Again, all of this
information should be easily available to the tester, without leaving the
desktop.

Testing, if done this way, would allow the tester to concentrate on the things
that really matter and that make testing a skilled activity; concentrate on the
hidden bugs that have not been found yet, on the special configurations and
environments, on generating new ways to break the software. On having fun while
testing.

\section*{Wrapping up}

From what I have seen in the latest three years, testing has improved a lot in
Ubuntu and the rest of Free Software projects that I am somehow involved with, but this
is not enough. If we really want to increase the quality Free Software
we need to start investing in testing and innovating the ways we do it, the same
way we invest in development. We cannot test 21st century software with 20th
century testing techniques. We need to react. Free Software is good because it is
open source is not enough anymore. Free Software will be good because it is open
source and has the best quality that we can offer.
