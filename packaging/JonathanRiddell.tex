\chapter{From Beginner to Professional - Jonathan Riddell}

\textit{Jonathan Riddell is a KDE and Kubuntu developer currently employed by
Canonical. When not at a computer he canoes the river of Scotland.}

There was a bug in the code. A nasty one too: a crash without saving data. That
is the problem with looking at code, you find things to fix. It is easy to get
involved in Free Software; the difficult part is getting out again. After the
first bug fix there are more and more, all within reach. Bug fixes lead to
adding features, which leads to project maintenance, which leads to running
community. 

It started with reading Slashdot, that mass of poorly filtered tech and geek
news with comments from anyone who can reload fast enough to get at the top.
Every news story was interesting and exciting, a fresh insight into the tech
world I was becoming fascinated with. No more did I have to accept what was
given to me by large software companies, here in the Free Software community I
could see the code develop in front of me.

As a university student it was possible to complete the exercises given by
lecturers very quickly, but exercises are not finished programs. I wanted to
know how to apply the simple skills they had given me to the real world by
writing programs which solve real problems for people. So I looked for the code,
which was not hard to find, just lying around on the Internet in fact. 
Looking closer at the code for the programs I was running I saw beauty. Not
because the code was perfectly tidy or well-structured, but because I could
understand it with the concepts I had already learned. Those classes, methods
and variables fell into place, enabling me to solve the relevant problems. Free
Software is the best way to make that step from knowing how to finish exercises
in a class to understanding how real programs get written.

Every computing student should work on Free Software for their dissertation.
Otherwise you get to spend six months to a year on a project only for it to sit
in the basement of a library never to be visited again. Only Free Software makes
it possible to excel by doing what comes naturally: wanting to learn how to
solve interesting problems. By the end of my project NASA programmers were using
my UML diagramming tool and it won awards with lavish receptions. With Free
Software you can solve real problems for real users.

The developer community is full of amazing people, with the passion and
dedication to work without any more reward than a successful computer program.
The user community is also awesome. It is satisfying to know you have helped
someone solve a problem, and I appreciate the thank you emails I receive.

Having written useful software, it needs to be made available to the masses.
Source code is not going to work for most people, it needs to be packaged up.
Before I was involved in it I looked down on packaging as a lazy way to
contribute to Free Software. You get to take much of the credit without having
to code anything. This is somewhat unfair, much of the community management
needed to run any Free Software project can also be seen as taking the credit
without doing the code.

Users depend on packagers a lot. It needs to be both fast, to keep those who
want the latest and greatest, and it needs to be reliable, for those who want
stability (which is everyone). The tricky part is that it involves working with
other people’s software, and other people’s software is always broken. Once
software is out in the wild problems start to emerge that were not visible on
the author’s own computer. Maybe the code does not compile with a different
compiler version, maybe the licensing is unclear so it can not be copied, maybe
the versioning is inconsistent so minor updates might be incompatible, screen
sizes might be different, desktop environments can affect it, sometimes
necessary third party libraries do not even have a release. These days software
needs to run on different architectures, 64-bit processors caused problems when
they became widely available, these days it is ARM which is defeating coders’
assumptions. Packagers need to deal with all of these issues, to give something
to the users which reliably works.

We have a policy in Ubuntu that packages with unit tests must have those tests
enabled as part of the package build process. Very often they fail and we get
told by the software author that the tests are only for his or her own use.
Unfortunately it is never reliable enough in software to test it yourself, it
needs others to test it too. One test is rarely enough, it needs a multi-layered
approach. The unit tests from the original program should be the first place to
start, then the packager tests it on his or her own computer, then it needs
others to test it too. Automatic install and upgrade testing can be scripted on
cloud computing services quite nicely. Putting it into the development
distribution archive gets wider testing before finally some months later it gets
released to the masses. At each stage problems can and will be found which need
to be fixed, then those fixes need testing. So there might not be much coding
involved but there is a lot of work to get the software from being 95\% to being
100\% ready, that 5\% is the hardest part, a slow and delicate process needing
careful attention all the way.

You can not do packaging without good communication with your upstream
developers. When bugs happen it is vital to be able to find the right person to
talk to quickly. It is important to get to know them well as friends and
colleagues. Conferences are vital for this as meeting someone gives much more
context to a mailing list post than a year of emails can. 

One of the unspoken parts of the Free Software world is the secret IRC channels
used by core members of a project. All big projects have them, somewhere out
there Linus Torvalds has a way of chatting to Andrew Morton et al about what is
good and what is bad in Linux. They are more social than technical and when
overused can be very anti-social for the community at large, but for those times
when there is a need for a quick communication channel without noise they work
well.

Blogging is another important method of communication in the Free Software
community. It is our main method of marketing and promotion for both the
software we produce and ourselves. Not to be used for shameless self-publicity,
there is no point claiming you will save lives with your blog, but used to talk
about your work on Free Software it builds community. It can even get you a job
or recognized in the street.

Those Slashdot stories of new technology developments are not about remote
figures you never meet in the way newspaper stories are. They are about people
who found a problem and solved it using the computer in front of them. For a few
years I was editing the KDE news site, finding the people who were solving
problems, creating novel ideas and doing the slow slog of getting the software
up to high enough quality, then telling the world about them. There were never a
shortage of people and stories to tell the world about. 

My last piece of advise is to stay varied. There is such a wealth of interesting
projects out there to explore, learn from and grow, but once in a position of
responsibility it can be tempting to stay there. Having helped create a
community for Kubuntu I am moving temporarily to work on Bazaar, a very
different project with a focus on developers rather than non-tech users. I can
start again learning how code turns into useful reality, how a community
interacts, how quality is maintained. It will be a fun challenge and I am
looking forward to it.
