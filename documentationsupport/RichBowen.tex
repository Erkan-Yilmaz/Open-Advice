\authorbio{Rich Bowen has been working on Free/Open Source Software for about 15 years.
Most of this time has been spent on the Apache HTTP Server, but he has also
worked on Perl, PHP, and a variety of web applications. He is the author of
Apache Cookbook, The Definitive Guide to Apache mod\_rewrite, and a variety of
other books, and makes frequent appearances at various technology conferences.}

\chapterwithauthor{Rich Bowen}{Good Manners Matter}

I started working on the Apache HTTP Server documentation project in September
of 2000. At least, that is when I made my first commit to the docs. Prior to
that, I had submitted a few patches via email, and someone else had committed
them.

Since that time, I have made a little over a thousand changes to the Apache HTTP
Server docs, along with just a handful of changes to the server code itself.

People get involved in Free/Open Source Software for a lot of different reasons.
Some are trying to make a name for themselves. Most are trying to ``scratch an
itch'', as the saying goes -- trying to get the software to do something that it
does not currently do, or create a new piece of software to fill a need that
they have.

I got involved in software documentation because I had been roped into helping
write a book, and the existing documentation was pretty awful. So, in order to
make the book coherent, I had to talk with various people on the project to help
make sense of the documentation. In the process of writing the book, I made the
documentation better, purely to make my work easier.

Around that same time, Larry Wall, the creator of the Perl programming language,
was promoting the idea that the three primary virtues of a programmer were
laziness, impatience and hubris. Larry was making very valid points, and Larry
has a sense of humor. A significant portion of the programmer community,
however, take his words as license to be jerks.

What I have learned over my years in Open Source documentation is that the three
primary virtues of a documentation specialist, and, more generally, of customer
support, are laziness, patience, and humility. And that the over-arching virtue
that ties these all together is respect.

\section*{Laziness}

We write documentation so that we do not have to answer the same questions every
day for the rest of our lives. If the documentation is inadequate, people will
have difficulty using the software. While this may be a recipe for a lucrative
consulting business, it is also a recipe for a short-lived software project, as
people will give up in frustration and move on to something that they do not
have to spend hours figuring out.

Thus, laziness is the first virtue of a documentation writer.

When a customer asks a question, we should answer that question thoroughly.
Exhaustively, even. We should then record that answer for posterity. We should
illuminate it with examples, and, if possible, diagrams and illustrations. We
should make sure that the prose is clear, grammatically correct, and eloquent.
We should then add this to the documentation in a place that is easy to find,
and copiously cross referenced from everywhere that someone might look for it.

The next time someone asks this same question, we can answer them with a pointer
to the answer. And questions that they may have after reading what has already
been written should be the source of enhancements and annotations to what has
already been written.

This is the true laziness. Laziness does not mean merely shirking work. It means
doing the work so well that it never has to be done again.

\section*{Patience}
There is a tendency in the technical documentation world to be impatient and
belligerent. The sources of this impatience are numerous. Some people feel that,
since they had to work hard to figure this stuff out, you should to. Many of us
in the technical world are self-taught, and we have very little patience for
people who come after us and want a quick road to success.

I like to refer to this as the ``get off my lawn'' attitude. It is not very
helpful.

If you cannot be patient with the customer, then you should not be involved in
customer support. If you find yourself getting angry when someone does not get
it, you should perhaps let someone else take the question.

Of course, that is very easy to say, and a lot harder to do. If you find
yourself in the position of being an expert on a particular subject, people are
inevitably going to come to you with their questions.
You are obliged to be patient, but how do you go about achieving this? That
comes with humility.

\section*{Humility}
I had been doing technical support, particularly on mailing lists, for about two
years, when I first started attending technical conferences. Those first few
years were a lot of fun. Idiots would come onto a mailing list, and ask a stupid
question that a thousand other losers had asked before them. If they had taken
even two minutes to just look, they would have found all the places the question
had been answered before. But they were too lazy and dumb to do that.

Then I attended a conference, and discovered a few things.

First, I discovered that the people asking these questions were people. They
were not merely a block of monospaced black text on a white background. They
were individuals. They had kids. They had hobbies. They knew so much more than I
did about a whole range of things. I met brilliant people for whom the
technology was a tool to accomplish something non-technical. They wanted to
share their recipes with other chefs. They wanted to help children in west
Africa learn how to read. They were passionate about wine, and wanted to learn
more. They were, in short, smarter than I am, and my arrogance was the only
thing between them and further success.

When I returned from that first conference, I saw the users mailing list in an
entirely different light. These were no longer idiots asking stupid questions.
These were people who needed just a little bit of my help so that they could get
a task done, but, for the most part, their passions were not technology.
Technology was just a tool. So if they did not spend hours reading last year’s
mailing list archives, and chose instead to ask the question afresh, that was
understandable.

And, surely, if on any given day it is irritating to have to help them, the
polite thing to do is to step back and let someone else handle the question,
rather than telling them what an imbecile they are. And, too, to remember all of
the times I have had to ask the stupid questions.

\section*{Politeness and Respect}
In the end, this all comes down to politeness and respect. Although I have
talked mainly here about technical support, the documentation is simply a static
form of technical support. It answers the questions that you expect people to
have, and it provides these answers in a semi-permanent form for reference.

When writing this documentation, you should attempt to strike the balance
between assuming that your reader is an idiot, and assuming that they should
already know everything. At the one end, you are telling them to make sure their
computer is plugged in. At the other end you are using words like ``simply'' and
``just'' to make it sound like every task is trivial, leaving the reader feeling
that they are probably not quite up to the task.

This involves having a great deal of respect and empathy for your reader, and
endeavoring to remember what it was like to be in the beginner and intermediate
stages of learning a new software package. Examples of bad documentation are so
prevalent, however, that this should not be a terribly difficult memory to
rekindle. Chances are that you have felt that way within the last week.

\section*{I wish ...}
I wish that when I started working on Open Source documentation I had been less
arrogant. I look back at some of the things that I have said on
publicly-archived mailing lists, forever enshrined on the Internet, and am
ashamed that I could be that rude.

The greatest human virtue is politeness. All other virtues flow from it. If you
cannot be polite, then all of the things that you accomplish amount to little.
