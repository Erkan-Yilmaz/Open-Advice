\chapter{Life Changer Documentation for Novice -- Atul Jha}

\textit{Atul Jha has been using Free Software since 1995. He is working as an
Application Specialist at CSS Corp, Chennai, India. He loves visiting colleges,
meeting students and spreading the word about Free Software.}

In 2005, the cyber cafe was the place to surf Internet as dial up connections
were very costly. During that time, Yahoo chat was very popular and I used to
visit the \#hackers channel there. There were some crazy people there who said they
would hack my password. I was very eager to know more about hacking and become
one of them. The next day I went to the cyber cafe again and typed ``how to become a
hacker'' on Yahoo search. The very first URL was of Eric S. Raymond. I was jumping
with joy that I had the magic key.
 
I started reading the book and to my surprise the definition of hacker was
``someone who likes solving problems and overcoming limits''. It also said
``hackers build things, crackers break them.'' Alas I wanted to be a cracker but
this book brought me to the other world of hacking. I kept reading the book and
encountered various new terms like GNU/Linux, mailing list, Linux user group,
IRC, Python and many more. 

After searching further, I was able to find a Linux user group in
Delhi and got a chance to meet real hackers. I felt like I was in an
alien world as they were talking about Perl, RMS, the kernel, device
drivers, compilation and many other things which were going over my
head.

I was in a different world. I preferred coming back home and finding some
Linux distribution from somewhere. I was too scared to ask for one
from them. I was nowhere near their level, a total dumb newbie. I managed to get some
distribution by paying 1000 Rs to a guy who used to have a business
selling distribution media. I tried many of them and was not able to
get my sound working. This time I decided to visit an IRC channel from
the cyber cafe. I found \#linux-india and jumped over asking ``my sound
nt wrking'', then came instructions like ``no SMS speak'' and ``RTFM''. It
scared me more and took some time to figure out that RTFM meant ``read the
f*** manual''.

I was terrified and preferred staying away from IRC for a few weeks.

One fine day I got an email about a monthly Linux user group meetup. I
needed answers for my many questions. I met Karunakar there and he
asked me to bring my computer to his office as he had the whole Debian
repository available there. Debian was new for me but I was
satisfied with the fact that finally I will be able to play music on
Linux. The next day I was in his office after carrying my computer on the
over-crowded bus -- it was fun. In a few hours, Debian was up and running on my
system. He also gave me a few books for beginners and a command reference.

The next day again in the cyber cafe, I read another of Eric S. Raymond's essay
called \textit{How To Ask Questions The Smart Way}. I was still visiting the \#hackers channel on
Yahoo chat where I made a very good friend, Krish, who suggested me to
buy a book called \textit{Linux Command Reference}. After spending some time
with those books and looking things up at tldp.org I was a newbie
Linux user. I never looked back. I also attended a Linux conference
where I met a few hackers from Yahoo and I was really inspired after
attending their talk. Later after a few years I had a chance to meet Richard Stallman
who is more like a god for many people in Free Software community.

I would admit that the documentation of Eric S. Raymond changed my life and that of
many others for sure. After all these years in the Free Software
community, I have realized documentation is the key for participation
of newbies in this awesome Open Source community. My 1\$ advice to all
developers would be to please document even the smallest work you do as the world
is full of newbies who would love to read it. My blog has even simple
postings like enabling the spell checker in OpenOffice to installing Django
in a virtual environment.
