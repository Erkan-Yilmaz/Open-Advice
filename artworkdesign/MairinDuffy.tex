\chapter{Don't Be Shy - Máirín Duffy Strode}

\textit{Máirín Duffy Strode has been using free \& open source software since she
was in high school, and has been a contributor for the past 8 years. She is involved
in both the Fedora and GNOME communities and has worked on interaction design,
branding, and/or iconography for a number of prominent FOSS applications such as
Spacewalk, Anaconda, virt-manager, SELinux, and SSSD. She has also been involved
in outreach efforts teaching children design skills using FOSS tools such as
GIMP and Inkscape and is a fierce advocate for said tools. She is the team lead
of the Fedora Design Team and a senior interaction designer with Red Hat, Inc.}

I knew about and used free and open source software for a long time before I
became a contributor. This was not for lack of trying -- there were a couple of
false starts, and I succumbed to them mostly out of being too shy and afraid to
push through them. From the aftermath of those false starts and also from
on-boarding other designers in FOSS projects, I have five tips to offer to you
as a designer trying to ramp up as a FOSS contributor:

\section*{1. Know that you are needed and wanted (badly!)}

My first false start happened when I was a first-year computer science student
at Rensselaer Polytechnic Institute. There was a particular project I used a lot
and I wanted to get involved with it. I did not know anyone in the project (or
anyone who was involved in free software) so I was trying to get involved pretty
cold. The project's website indicated that they wanted help and that they had an
IRC channel, so I lurked in there for a week or two. One day after a lull in
conversation, I spoke up: I said I was a computer science student interested in
usability and that I would love to get involved.

``Go away'' was the response. Furthermore, I was told that \emph{my} help was not
needed nor wanted. 

This set me back a few years in getting involved -- just a few harsh words on IRC
made me afraid to try again for almost 5 years.  I did not discover until much
later that the person who had essentially chased me out of that project's IRC
channel was on the fringes of the project and had a long history of such
behavior, and that I really had not done anything wrong. If I had only kept
trying and talked to other people, I may have been able to get started back
then.

If you would like to contribute to free and open source software I guarantee you
there is a project out there that really needs your help, especially if you are
design-minded! Are you into web design? Iconography? Usability? Skinning? UI
mockups? I have spoken to many FOSS developers who are not only desperate for
this kind of help, but who would also deeply appreciate it and love you to
pieces for providing it.

If you encounter some initial resistance when first trying to get started with a
project, learn from my experience and do not give up right away. If that project
turns out to not be right for you, though, do not worry and move on. Chances are,
you are going to find a project you love that loves you back.

\section*{2. Help the project help you help them}

Many free \& open source software projects today are dominated by programmers and
engineers and while some are lucky enough to have the involvement of a creative
person or two, for most projects a designer, artist, or other creative's
presence is an often-yearned-for-yet-never-realized dream. In other words, even
though they understand they need your skills, they may not know what kinds of
help they can ask you for, what information they need to give you to be
productive, or even the basics of how to work with you effectively. 

When I first started getting involved in various FOSS projects, I encountered
many developers who had never worked directly with a designer before. At first,
I felt pretty useless. I could not follow all of their conversation on IRC
because they involved technical details about backend pieces I was not familiar
with. When they bothered to pay attention to me, they asked questions like,
``What color should I put here?'' or ``What font should I use?'' What I really
wanted as an interaction designer was to be privy to decision-making about how
to approach the requirements for the project. If a user needed a particular
feature, I wanted to have a say in its design -- but I did not know when or where
those decisions were happening and I felt shut out.

Design contains a pretty wide range of skills (illustration, typography,
interaction design, visual design, icon design, graphic design, wordsmithing,
etc.) and any given designer likely does not possess all of them. It is
understandable, then, that a developer might not be sure what to ask you for.
It is not that they are trying to shut you out -- they just do not know how you
need or want to be involved.

Help them help you. Make it clear to them the kind of work you would like to offer
by providing samples of other
work you have done. Let them know what you need so they can better understand how
to help you engage in their project. For example -- when you first get involved
in a particular initiative for the project, take the time to outline the design
process for it, and post it on the main development list so other contributors
can follow along. If you need input at particular points in the process, flag
those points in your outline. If you are not sure how particular things happen --
such as the process for developing a new feature -- approach someone on the side
and ask them to walk you through it. If someone asks you to do something beyond
your technical ability -- working with version-control, for example -- and you are
not comfortable with that, say so.

Communicating your process and needs will prevent the project from having to
make guesses and instead they will be able to make the best use of your talents.

\section*{3. Ask questions. Lots of questions. There are no stupid questions.}

We have noticed sometimes in Fedora that when new designers come on board, they
are afraid to ask technical questions for fear they will look stupid. 

The secret is, developers can be so specialized that there are a lot of
technical details outside of their immediate expertise that they do not
understand either -- this happens even within the same project. The difference is
that they are not afraid to ask -- so you should not be, either! In my interaction
design work, for example, I have had to approach multiple folks on the same
project to understand how a particular workflow in the software happens, because
it is passed off between a number of subsystems and not every person in the
project understands how every subsystem works. 

If you are not sure what to work on, or you are not sure how to get started, or
you are not sure why that thing someone said in chat is so funny -- ask. It is a
lot more likely someone is going to tell you that they do not know either, than
they are going to think that you are stupid. In most cases, you will learn
something new that will help make you a better contributor.

It can be especially effective to seek out a mentor -- some projects even have
mentoring programs -- and ask them if they would not mind being your go-to person
when you have questions. 

\section*{4. Share and share often. Even if it is not ready yet. Especially if it
is not ready yet.}

We have also noticed new designers in Fedora and other free \& open source projects
are a little shy when it comes to showing their work. I understand that you
do not want to ruin your reputation by putting something out there that is not
your best or even finished, but a big part of how free \& open source projects
work is sharing often and openly. 

The further along you have come on a piece before you have shared it, the harder
others will find it to provide you actionable feedback and to jump in and get
involved. It is also harder for others to collaborate on your piece themselves
and feel a sense of ownership for it, supporting and championing it through to
implementation. In some free \& open source projects, not being forthcoming with
your sketches, designs, and ideas is even seen as offensive! 

Post your ideas, mockups, or designs on the web rather than in email, so it is
easy for others in the project to refer to your asset via copying \& pasting the
URL -- especially handy during discussions. The easier it is to find your design
assets, the more likely it is they will be used. 

Give this tip a try and keep an open mind. Share your work early and often, and
make your source files available. You might be pleasantly surprised by what
happens!

\section*{5 Be as visible as you can within the project community.}

One tool that -- completely unintentionally -- ended up helping me immensely in
getting started as an open source contributor was my blog. I started keeping a
blog, just for myself, as a sort of rough portfolio of the things I had been
working on. My blog is a huge asset for me, because:
\begin{itemize}
 \item As a historical record of project decisions, it is a convenient way to look up
old design decisions -- figure out why we had decided to drop that screen again,
or why a particular approach we had tried before did not work out, for example.
 \item As a communication device, it helps other contributors associated with your
project and even users become aware of what work is happening and aware of
upcoming changes in the project. Many times I have missed something essential in a
design, and these folks have been very quick to post a comment letting me know!
 \item It helped me to build my reputation as a free software designer, which has
helped me build others' trust in my design decisions as time has gone on. 
\end{itemize}

Do you blog? Find out which blog aggregations the members of the free \& open
source project you are working on read, and put in requests to have your blog
added to them (there is usually a link to do so in the sidebar.) For example, the
main blog aggregator you will want to join to become a part of the Fedora
community is called Planet Fedora\footnote{\url{http://planet.fedoraproject.org}}. Write a first blog
post once you have been added introducing yourself and letting folks know what you
like -- all of the sort of information advised in tip \#1.

The project will surely have a mailing list or forum where discussion takes
place. Join it, and send an intro there too. When you create assets for the
project -- no matter how small, no matter how unfinished -- blog about them,
upload them to the project wiki, tweet/dent about them, and send links to
prominent community members on IRC to get their feedback.

Make your work visible, and folks will start to associate you with your work and
approach you with cool projects and other opportunities based solely on that.


This is everything I wish I had known when first trying to get involved in free
\& open source software as a designer. If there is any one thing you should
take away from this, it is that you should not be shy -- please speak up, please
let your needs be known, please let others know about your talents so they can
help you apply them to making free software rock.
