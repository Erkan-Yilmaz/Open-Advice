\chapter{Building Bridges -- Shane Coughlan}

\textit{Shane Coughlan is an expert in communication methods and business
development. He is best known for building bridges between commercial and
non-commercial stakeholders in the technology sector. His professional
accomplishments include establishing a legal department for the main NGO
promoting Free Software in Europe, building a professional network of over 270
legal and technical experts across 27 countries, co-founding a binary code
compliance tool project and aligning corporate and community interests to launch
the first law review dedicated to Free/Open Source Software. Shane has extensive
knowledge of Internet technologies, management best practice, community building
and Free/Open Source Software. He has delivered speeches and lead panels at
dozens of events in Europe, the USA and across Asia.}

When I started to work in Free Software I was struck by the perceived difference
between the ``community'' and the ``business'' stakeholders in this field. The
informal assertion often aired at the time essentially proposed that there were
developers interested in hacking and there were commercial parties who would use
their output in objectionable ways if not closely monitored. It was a generally
baseless assumption, and almost entirely limited to parties who identified
themselves as the community rather than those more aligned with business
interests, but it was prevalent.

Despite being primarily associated with the community side of things, I resisted
the concept that there were two inherently hostile parties facing each other
down over the future of Free Software. It sounded too simple to frame the
dynamics of contribution, use and support as the interplay between noble
creators and devious freeloaders. Indeed, it sounded more like a situation where
complexity, change and uncertainty had lead to the creation of simplistic
narratives to provide comfort for parties moving out of their comfort zone. I
could feel the tension in the air, I could hear the arguments at booths and in
meetings, and I could observe the sharp comments or blowing off of steam at
conferences. But what did it all mean?

Whether we were talking about Free Software project contribution, project
management or license compliance, the relationships between stakeholders were
often accompanied by assumptions, lack of communication and negative emotion.
This in turn lead to greater complexity and a corresponding increase in the
difficulty of making unified decisions or resolving issues. I was aware that one
of the biggest challenges was how to build bridges between individuals, projects
and businesses, a necessary step to ensure common understanding and
cross-communication of the rules, norms and reasons behind the licenses and
other formal measures to govern this field, but that in itself does not
translate into knowing how to engage with the issue effectively.

This was at the tipping point when GPLv3 was being drafted, Linux-based
technology was beginning to appear in all sorts of consumer electronics, and
Free Software was at the brink of becoming mainstream. Change was in the air and
business investment around major Free Software projects was spiking. Suddenly
there were major corporation employees actually doing a lot of the difficult
work, there was significant funding available for events, and a lot of the
software stopped being about fun, and started to be about milestones,
deliverables, quality assurance and usability.

This was probably a system shock to parties who had been doing Free Software for
a long time. For much of its evolution Free Software was not just about
technical exploration and perfection, but also social interaction. It provided a
way for intelligent though occasionally awkward people to share a common
interest, to challenge each other, and to cooperate inside carefully delineated
and predictable lines. Like stamp collecting, train spotting or Star Trek, it
was a place where detail-orientated people could converge, and it had the
additional benefit of providing broader feel-good social benefits as an output.
It was not where the original contributors had expected to encounter
middle-management and output-orientated development focus. No wonder a few noses
were out of joint.

And yet \dots Everything worked out fine. Free Software is everywhere, and
appears to be in an almost unassailable position as a mainstream component of
the Information Technology industry. Projects like the Linux kernel or the
Apache server have continued to grow, to innovate and to attract new
stakeholders, both commercial and non-commercial. The balance of power between
individuals, projects and businesses changed, occasionally with conflict and
disruption, but never at the cost of long-term cooperation or of undermining the
core value of Free Software.

From my perspective in the legal field -- which after all is merely a formal
language that provides a context for interaction through mutually understood and
enforceable rules -- the tension in Free Software did not lie in the
introduction of increased commercial activity, in the increased participation of
company employees in projects, or in change itself. The real problem lay in the
gap between a displaced previous elite and their newer, occasionally very
different, fellow stakeholders.

The challenge was to create a level playing field where the different interests
could co-exist with mutual respect. Free Software needed to become a place where
information like the proper remit and obligations of a license or requirements
for code submissions to a project could be obtained by any party at any time.
Subjectivity and vagueness needed to take a backseat to allow the formation of
more formalized transactions, which in turn act as an essential precursor for
any large economic activity, especially in the context of an international or
global community.

What had worked in the early days -- be it the trust of a few parties or the
common understanding reached by a similar group with similar interests -- could
no longer act as social or economic drivers for the future of the field. At
times this seemed like an insurmountable barrier and that the tensions between
the previous contributors to Free Software and new stakeholders must lead to a
collapse of cooperation and perhaps of the progress made. But such a grim
outcome would presuppose conditions that simply did not exist.

Free Software provided a lot of value to different people and organizations
based on some very simple concepts like the freedom to use, modify, improve and
share technology. These concepts allowed a great deal of flexibility, and as
long as people recognized their value and continued to respect them, challenges
over secondary items like project governance or license gray areas were -- in
the long run -- pretty much irrelevant. The rest was mainly noise, the normal
communication spike with all its trappings of drama that inevitably occurs when
one social group is joined by another. The same applies whether we are talking
about a fishing spot, a country welcoming immigrants (or not), or two businesses
merging.

The changes in Free Software all looked a little confusing at the time, but
essentially break down into three useful lessons that will be familiar to
students of history or political science. Firstly, whenever there is an elite,
it will seek to preserve its status and it will communicate the perceived
challenge as a negative development in an attempt to undermine it. Secondly,
despite the inherent tendency of any power base to be conservative, static
engagement with a changing field will only result in moving the opportunity for
improvement from existing parties to third parties. Finally, if something has
value, then challenges in governance are unlikely to undermine that value, but
instead will provide a method of refining both the governance mechanisms and the
people in a position to apply them.

The development of Free Software as a mainstream technology saw increased
professionalization in both the approach of developers and in the management of
projects. It also saw greater respect for licenses on the part of individuals,
projects and companies. This was no bad thing, and despite a few rocky moments
along the way -- you can take your pick from inter-community fighting, companies
disregarding license terms or the upset caused by a move away from beer and
t-shirt culture -- we are left with a stronger, more coherent and more valuable
field.
