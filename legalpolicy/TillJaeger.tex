\chapter{I wish I would have known when I started... - Dr. Till Jaeger} 

\textit{Dr. Till Jaeger has been a partner at JBB Rechtsanwaelte since 2001. He is a Certified Copyright and Media Law Attorney and advises large and medium-sized IT businesses as well as government authorities and software developers on matters involving contracts, licensing and online use. One particular focus of his work is on the legal issues created by Free and Open Source Software. He is co-founder of the Institute for Legal Aspects of Free \& Open Source Software (ifrOSS), contributing to its work with academic publications, lectures and seminars in the fields of software law and copyright law. He provides advice on compliance with open source licenses and on compatibility issues, and helps developers and software companies to enforce licenses both in Germany and beyond.
\newline
Till is a member of the Institute for Legal Questions on Free and Open Source Software (ifrOSS) in Germany which he co-founded in 2000 to perform research on legal questions regarding the Open Source model. ifrOSS was involved in several legislative projects on the national and European level, such as the implementation of the EU Information Society Directive in Germany, the German Copyright License Act and the discussion on the proposal for a directive on the patentability of computer-implemented inventions.
\newline
Till represented the gpl-violations.org project in several lawsuits to enforce the GPL and has published several articles and books related to legal questions of Free and Open Source Software. He was a member of the Committee C in the GPLv3 drafting process.}

One thing upfront: I am not a geek. I never have been one, and have no intention to become one in the future. 
Instead, I am a lawyer. Most people who read this book probably tend to sympathize more with geeks than with lawyers. Nevertheless, I do not want to hide this fact. That the FOSS community is not necessarily fond of lawyers but busy developing software is something I \textit{did} know about FOSS in early 1999, when our ways first crossed. And there were quite a few things I did not know.
In 1999, while completing my doctoral thesis that focused on a classical copyright topic, I was assessing the scope of moral rights. In this context I spent a while pondering about the question of how moral rights of programmers are safeguarded by the GPL, which allows others to modify their programs. This is how I first got in contact with FOSS, whereas at the time, “free” and “open” certainly had different meanings not worth to argue about in the world I was living in; since I was free to do what I was interested in and open to investigate new copyright questions, I soon found out that the two words \textit{do} have something in common, yet that they are \textit{different}, yet are best used together...
What I wish to have also already known back then are three things:
First, my technical knowledge, most relevantly in the field of software, was insufficient. Second, I did not really know the community and what mattered to the people who are part of it. Last but not least, I did not know much about foreign jurisdictions back then. It would have been useful to know all that from the beginning. Meanwhile, I have learned a fair bit, and just as the community is happy to share its achievements I am happy to share my lessons\footnote{The “Institut f\"ur Rechtsfragen der Freien und Open Source Software” (Institute for Legal Questions on Free and Open Source Software) offers, inter alia, a collection of FOSS related literature and court decisions; see www.ifross.org for details.}:

\paragraph*{Technical knowledge}
How is software architecture shaped? What is the technical structure of software like? Which licenses are compatible with each other and which are not, and how and why? How is the Linux kernel structured? 
To name one example, the important question of what constitutes a “derivative work” according to the GPL determines how the software may be licensed. Everything that counts as derived from GPL-licensed software must be distributed under the GPL. To assess whether a certain software is a “derivative work” or not requires profound technical understanding, since the interaction of program modules, like linking, IPC, plugins, framework technology, header files and so on determines among other criteria, whether a program is formally inseparable, which reflects on finding it being derived from another program or not. 

\paragraph*{Knowledge of the industry and the community}
Besides these functionality issues I had no profound understanding of the idea behind FOSS and the motivation of the developers and the companies that use FOSS. Neither did I really know about its philosophical background, nor was I familiar with practical issues such as “who is a maintainer?” or “how do version control systems work?” In order to serve your clients best, these matters are no less important than your proficiency in technical aspects.  
Our clients ask us about legal aspects of forming business models such as dual licensing, “open core”, support and services contracts, code development and code contribution agreements. We consult clients concerning what FOSS might have in store for their companies or institutions. We also advise developers on what they can do about infringements of their copyrights, and draft and negotiate contracts for them. In order to serve such clients comprehensively, it is important to be familiar with the different points of view.  

\paragraph*{Comparative Law knowledge}
The third thing a FOSS lawyer needs is knowledge about foreign jurisdictions, at least a few aspects of it, and the more the better. In order to construe the different licenses, it is essential to be familiar with the perspective of the people who have drafted it. In most cases the U.S. legal system is of key importance. For example, the GPL was drafted with U.S. legal concepts in mind. In the United States, “distribution” includes online distribution, whereas under the German Copyright system there is a distinction between offline and online distribution. Licenses that have been drafted by lawyers from the United States may thus be construed as including online distribution, which might be relevant and helpful in court proceedings\footnote{Cf. Welte v. Skype, 2007,  http://www.ifross.org/Fremdartikel/LGMuenchenUrteil.pdf}.
 
So, all this is useful to know. And as software keeps on being developed and modified to provide answers to the needs of the day, so my mind will hopefully keep on finding answers to the challenges the vibrant FOSS community poses to a lawyers mind.