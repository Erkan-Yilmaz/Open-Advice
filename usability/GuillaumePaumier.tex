\chapter{Learn from your users - Guillaume Paumier}

\textit{Guillaume Paumier is a photographer and physicist living in Toulouse,
France. A long-time Wikipedian, he is currently working for the Wikimedia
Foundation, the non-profit that runs Wikipedia. As a product manager for
Multimedia Usability, he notably conducted user research to design a new media
upload system for Wikimedia Commons, the free media library associated with
Wikipedia.}

You know Wikipedia, the freely reusable encyclopedia that anyone can edit. It
was created in 2001 and recently celebrated its tenth anniversary. Despite being
one of the ten most visited websites in the world, its user interface still
looks very ``1.0'', compared to what interactive web technologies allow. Some
might say it's for the best: Wikipedia is ``serious stuff'', and the user
shouldn't be distracted by fireworks in the interface. Yet, Wikipedia has had
issues recruiting new participants in the last few years, in part because of its
interface, that some may call archaic. This might explain why surveys of
Wikipedia participants have repeatedly shown a bias towards young,
technology-savvy men, many with a background in computers and engineering.
Besides the fact that free knowledge and free licenses sprouted from the fertile
land of Free and Open Source Software, the complicated interface has discouraged
many motivated potential participants.

In 2011, when all the major online publishing and collaboration platforms (like
WordPress, Etherpad and Google Documents) offer a visual editor to some extent,
Wikipedia is still using a many-year-old default wikitext editor that uses
quotes (\textquotesingle \textquotesingle \textquotesingle \textquotesingle) and
brackets ([[]]) for formatting. Efforts are underway to transition to a default
visual editor in 2012, but it isn't an easy challenge to solve.

But let us put the editor aside for a moment. The interface of Wikipedia remains
fairly complicated, and many useful features are difficult to discover. Did you
know Wikipedia had an integrated version control system, and you could see all
the previous versions of a page? Did you know you could see the list of all the
edits made by a participant? Did you know you could link to a specific version
of a page? Did you know you could export a page to PDF, or create custom
hardcover books from Wikipedia content, to be sent to your home?

Most Wikipedia readers arrive through search engines. Statistics show they spend
little time on Wikipedia, once they find the information they were looking for.
Few stick around and explore what tools the interface offers. For example,
Wikipedia is routinely criticized about its quality and reliability. Many of
these unexplored, almost hidden tools could prove useful to readers to help them
assess the reliability of information.

Wikipedia and its sister projects are powered by a wiki engine called MediaWiki
(and supported by the Wikimedia Foundation; all these confusing names alone are
a usability sin). For a long time, the development of MediaWiki was primarily
led by software developers. The MediaWiki community has a strong developer base;
actually, this community is almost entirely composed of developers. Only
recently did designers join the community, and they were hired by the Wikimedia
Foundation in this capacity. There are hardly any volunteer designers in the
community. This has caused the application to be built and ``designed''
exclusively by developers. As a consequence, the interface has naturally taken a
shape that closely follows the ``implementation model'', i.e., the way the
software is implemented in the code and data structures. Only rarely does this
implementation model match the ``user model'', i.e., the way the user imagines
things to work.

It would be unfair to say that developers do not care about users. The whole
point of creating software (apart from the sheer pleasure of learning stuff,
writing code and solving problems) is to release it so it can be used. This is
particularly true in the world of Free and Open Source Software, where most
developers selflessly volunteer their time and expertise. One might argue that
many developers are, in fact, users of their own product, especially in the
world of Free and Open Source software. After all, they created it, or joined
its team, for a reason, and this reason was rarely money. As a consequence,
developers of this kind of software would be in an ideal position to know what
the user wants.
\newline
But let's face it: you are not your regular user.
\newline
It is particularly hard for the developer to sit in the user's chair. For one
thing, your familiarity with the code and the software's implementation makes
you see its features and interface from a very specific, omniscient angle. You
know each and every feature of the application you created. You know where to
find everything. If something with the interface feels a little odd, you may
unconsciously discard it because you know it is a side-effect of how you
implemented such or such feature.

Let us say you are creating an application that stores data in tabular form
(possibly in a database). When the time comes to show this data to the user, you
will naturally think of the data as tabular, because it is how you implemented
it. It will make sense to you to display it in a way that is consistent with how
it is stored. Similarly, any kind of array or other sequential structure is
bound to be remembered as such, and displayed in a sequential format in the
interface as well, perhaps as a list. However, another format may make more
sense for the regular user, for example a set of sentences, a chart, or another
visual representation.

Another challenge is the level of expertise of developers. Because you want your
application to be awesome, you are likely to do a lot of research to build it.
In the end, you may not only become an expert in your application, but also an
expert in your application's topic. Many of your users will not have (or need)
that level of expertise, and they may be lost with the level of detail of some
features, or be unfamiliar with some terms the layperson does not know.

So, what can you do to fix it? Watch users. Seriously. Watching people as they
use your application is truly an eye-opening experience.

Now, one way to watch people use your application is to hire a usability firm,
who will recruit testers with various profiles among a pool of thousands,
prepare an interview script, rent a room in a usability lab with a
screen-recording app, a video camera pointed at the tester, and you in a
backroom behind a one-way glass, head-desking and swearing every time the user
does something you think does not make any sense. If you can afford to do that,
then by all means, do so. What you will learn will really change your
perspective. If you can not afford professional testing, all is not lost; you
are just going to have to do it yourself. Just sit beside a user as they show
you how they perform their tasks and go through their workflow. Be a silent
observer: your goal is to observe, and note everything. Many things will
surprise you. Once the user is done, you can go through your notes and ask
questions to help you understand how they think. To know more about
do-it-yourself testing have a look at \textit{Don't Make Me Think: A Common
Sense Approach to Web Usability} by Steve Krug, \textit{About Face 3: The
Essentials of Interaction Design} by Alan Cooper, Robert Reimann and David
Cronin and the OpenUsability project\footnote{\url{http://openusability.org}}.
It can be a bit awkward for users to be watched, yet I bet many of them will
happily volunteer to help you improve your application. Users who cannot
contribute code are usually happy to find other ways to participate, and showing
you how they use the software is a very easy way to do so. Users are generally
grateful for the time you have spent developing the application, and they want
to give back.

You will need to keep in mind, though, that not everything your users request
can or should be done. Listen carefully to their stories: it is an opportunity
for you to identify issues. But because a user requests a feature does not mean
they really need that feature; maybe the best way to fix the issue underlying
their feature request is to implement a completely different feature. Do not
listen blindly to your users. But you probably knew that already. 
\newline
Oh, and by the way, do not ask your family, either.
\newline
No offense intended, I am sure your mom and dad and sisters and brothers are very nice people. But if you are creating an accounting application, and your sister has never done any accounting, she is going to be quite lost. You will spend more time explaining what double-entry bookkeeping is, than really testing your software. However, your mom, who bought herself a digital camera last year, can be an ideal tester if you are creating an application to manage digital photos, or to upload them to a popular online sharing platform. For your accounting application, you can ask one of your colleagues or friends who already knows a thing or two about accounting.
\newline
Ask different people, too.
\newline
For some cosmological reason, people will find endless ways to use and abuse
your application, and break it in ways you would not think of in your worst
nightmares. Some will implement processes and workflows with your application
that make absolutely no sense to you, and you will want to slam your head on
your desk. Others will use your application in ways so smart, they will make you
feel stupid. Try to listen to users with different profiles, who have different
goals when they use your application.
\newline
Users are an unpredictable species. But they are on your side. Learn from them.

\section*{If you remember nothing else, ...}
... then remember this:
\begin{itemize}
 \item You will be tempted to make the interface look and behave like how it
works in the back-end. Your users can help you prevent that.
 \item Users are an unpredictable species. They will break, abuse and optimize
your application in ways you can not even imagine.
 \item Learn from your users. Improve your application based on what you
learned. Profit.
\end{itemize}
