\chapter{Being Allowed To Do Awesome - Lydia Pintscher}

\textit{Lydia Pintscher is a people geek and cat herder by nature. Among other
things, she manages KDE's mentoring programs (Google Summer of Code, Google
Code-in, Season of KDE) and is a founding member of KDE's Community Working
Group.}

Free Software has an enemy. It is not who most people on the Internet think it
is. No, it is a lack of active participation.

Every single day there are thousands of people out there looking for a way to
put meaning into their life, looking for ways to do something that truly
matters. Every single day thousands of lines of code for Free Software projects
are waiting to be written and debugged, programs are waiting to be promoted and
translated, artwork is waiting to be created and much more. Sadly, far too often
the people fail to connect with projects. There are several reasons for
that. It starts with people not knowing about Free Software at all and its
benefits and purpose. But we are getting there. People are starting to use and
maybe even understand Free Software on a large scale. Free Software projects
live by converting some of those users into active contributors. This is where
the problems begin.

I have managed hundreds of students in mentoring programs and have been doing
outreach in various forms for Free Software projects. I've worked with
enthusiastic people whose life was changed for the better by their contributions
to Free Software. But there is one theme I see over and over again and it breaks
my heart because I now know what talent we are missing out on: not being allowed
to do awesome. It is best summarized by what a fellow Google Summer of Code
mentor said: ``The insight that most people in Open Source didn’t get allowed to
work on stuff but just didn’t run fast enough at the right moment seems to be
rare''. Potential contributors often think they are not allowed to contribute.
The reasons for this are many and they are all misconceptions. The most common
misconceptions in my experience are:
\begin{itemize}
 \item ``I can not write code. There can not possibly be a way for me to
contribute.''
 \item ``I am not really good at this. My help is not needed.''
 \item ``I would just be a burden. They have more important things to worry
about.''
 \item ``I am not needed. They must already have enough much more brilliant
people than me.''
\end{itemize}
Those are almost always false and I wish I had known a long time ago that they
are so prevalent. I would have done a lot of my initial outreach efforts
differently.

The easiest way of getting someone out of this situation is to invite
them personally. ``That workshop we are doing? Oh yes, you should
come.'' ``That bug in the bug tracker? I'm sure you're the perfect
person to try to fix it.'' ``That press release we need to get done?
It would be great if you could read over it and make sure it is
good.'' And if that is not possible, make sure that your outreach
material (you have some, right?) clearly states what kind of people
you are looking for and what you consider the basic requirements. Make
sure to especially reach out to people outside your usual contributor
base because for them this barrier is even bigger. And unless you
overcome that, you will only recruit who you are -- meaning you will
get more contributors just like the ones you already have. People like
the people you already have are great, but think about all the other
great people you are missing out on, who could bring new ideas and
skills to your project.
