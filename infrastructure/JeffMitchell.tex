\chapter{Jeff Mitchell}
\todo{bio}
\todo{title}
\todo{Some paragraphs are very long; splitting them would improve readability}

A few days ago, I was part of a group interviewing a potential new sysadmin at
work. We had gone through a few dozen resumes and had finally brought our first
candidate in for an interview. The candidate -- let’s call him John -- had
experience with smaller, lab-style computer clusters as well as larger data
center operations. At first, things were proceeding apace, except that he had an
odd answer to a few of our questions: ``I’m a sysadmin.''
The meaning of that statement was not immediately clear to us, until the
following exchange occurred:
\begin{quote}
\textbf{Me}: So you’ve said that you don’t have Cisco IOS experience, but what about
networking in general?\newline
\textbf{John}: Well, I’m a sysadmin.\newline
\textbf{Me}: Right, but -- how about networking concepts? Routing protocols like BGP or
OSPF, VLANs, bridges \dots \newline
\textbf{John, exasperated}: I’m a sysadmin.
\end{quote}
That was when the light bulb came on. John had not been telling us that he knew
of the various things we were asking about because he is a sysadmin; he was
telling us that because he was a sysadmin he did not know about those things.
John was a systems administrator; claiming such was his hand-waving way of
indicating that those tasks belonged to network administrators.
John did not get the job.

For many open source projects, specialization is a curse, not a blessing.
Whether a project falls into one category or another often depends on the size
of the development team; specialization to the degree of single points of
failure can mean serious disruption to a project in the event of a developer
leaving, whether on good, bad or unfortunate terms. It is no different for open
source project sysadmins, although the general scarcity of these seems to allow
projects to adopt sometimes dangerous tolerances.
The most ridiculous example I have seen involved one particular project whose
documentation site (including all of their installation and configuration
documentation) was down for over a month. The reason? The server had crashed,
and the only person with access to that server was sailing around on a pirate
ship with members of Sweden’s Pirate Party. That really happened.
However, not all single points of failure are due to absentee system
administrators; some are artificial. One large project’s system administration
access rights decisions were handled by a single lead administrator, who not
only reserved some access rights solely for himself (yes, he did disappear for a
while and yes, that did cause problems) but made decisions about how access
rights should be given out based on whether he personally trusted the candidate.
``Trust'' in this case was based on one thing; it was not based on how many
community members vouched for that person, how long that person had been an
active and trusted contributor to that project, or even how long he had known
that individual as a part of that project. Rather, it was based on how well he
personally knew someone, by which he meant how well he knew that individual in
person. Imagine how well that scales to a distributed global team of system
administrators.
Of course, this example only goes to show that it is very difficult for open
source sysadmins to walk the line between security and capability. Large
corporations can afford redundant staff, even when those staff are segmented
into different responsibilities or security domains. Redundancy is important,
but what if the only current option for redundant system administration is
taking the first guy that randomly pops into your IRC channel and volunteers to
help? How can you reasonably trust that person, their skills, or their motives?
Unfortunately, only the project’s contributors, or some subset of them, can
determine when the right person has come along. The universe of open source
projects, their needs, and those willing to contribute to any particular project
is blissfully diverse; as a result, human dynamics, trust, intuition, and how to
apply them to any particular open source project are broad topics that are far
out of scope of this short essay.

One key thing has made walking that security/capability line far easier,
however: the rise of distributed version control systems, or DVCSes.
In the past, access control was paramount because the heart of any open source
project -- its source code -- was centralized. I realize that many out there
will now be thinking ``Jeff, you should know better than that; the heart of a
project is its community, not its code!'' My response is simple: community
members come and go, but if someone accidentally runs ``rm -rf'' on the entire
centralized VCS tree of your project, how many of those community members are
going to be willing to stick around and help recreate everything from scratch? A
project’s code is its heart; its community members are its lifeblood. Without
either, you are going to have a hard time keeping a project alive. Maybe you will
be lucky and be able to cobble together the entire source tree from checkouts
that different people had of different parts of the tree, but you will still have
lost all of your history, which for many projects is nearly as important as the
current code.
That is no longer the case. When every local clone has all of the history for a
project and nightly backups can be performed by having a cron job run e.g. ``git
pull'', the centralized repository is now just a coordination tool. This takes
its status down a few notches. It still has to be protected against threats both
internal and external: unpatched systems are still vulnerable to known exploits,
a malicious sysadmin can still wreak havoc, an ineffective authentication system
can allow malicious code into your codebase, and an accidental ``rm -rf'' of the
centralized repository can still cause loss of developer time. But these
challenges can be overcome, and in the day and age of cheap VPS and data center
hosting, absentee sysadmins can be overcome too. (Better make sure you have
redundant access to DNS, though! Oh, and, put your websites in a DVCS repository
too, and make branches for local modifications. You will thank me later.)
So, DVCSes give your project redundant hearts nearly for free, which is a great
way to help open source sysadmins sleep at night and makes us all feel a little
bit more like Time Lords. It also means if you are not on a DVCS, stop reading
this very moment and go get on one. It is not just about workflows and tools. If
you care about the safety of your code and your project, you will switch.

Source code redundancy is a must, and in general the greater amount of
redundancy you can manage, the more robust your systems. It may also seem
obvious that you want sysadmin redundancy; what you may not find obvious is that
redundant sysadmins are not as important as redundant skillsets. John, the
systems administrator, worked in data centers and companies with redundant
sysadmins but rigid, defined skillsets. While that worked for large companies
that could pay to acquire new sysadmins with particular skillsets on-demand,
most open source projects do not have that luxury. You have to make do with what
you can get. This of course means that an alternative (and sometimes the only
alternative) to finding redundant system administrators is spreading the load,
having other project members each pick up a skill or two until redundancy is
achieved.
It is no different from the developer or artwork side of a project; if half of
your application is written in C++ and half is written in Python, and only one
developer knows Python, a departure from the project by that developer will
cause massive short-term problems and could cause serious long-term problems as
well. Encouraging developers to branch out and become familiar with more
languages, paradigms, libraries, and so on means that each of your developers
becomes more valuable, which should not come as a shock; acquiring new skillsets
is a byproduct of further education, and more educated personnel are more
valuable.
Most open source developers that I know find it a challenge and a pleasure to
keep testing new waters, as that is the behavior that led them to open source
development in the first place. Similarly, open source system administrators are
in scarce supply, and can not afford to get stuck in a rut. New technologies
relevant to the sysadmin are always emerging, and there are often ways to use
existing or older technologies in novel ways to enhance infrastructure or
improve efficiency.
John was not a good candidate because he brought little value; he brought little
value because he had never pushed outside of his defined role. Open source
sysadmins falling into that trap do not just hurt the project they are currently
involved with, they reduce their value to other projects using different
infrastructure technologies that could desperately use a hand; this decreases
the overall capability of the open source community. To the successful open
source administrator, there is no such thing as a comfort zone.
 
\todo{The end is a bit abrupt}
