\chapter{Who are You, What are You Selling, and Why Should I Care? - Sally Khudairi}

\textit{Active in the Web since 1993, Sally Khudairi is the publicist behind
some of the industry's most prominent standards and organizations. The former
deputy to Sir Tim Berners-Lee and long-time champion of collaborative
innovation, she helped launch The Apache Software Foundation in 1999, and was
elected its first female and non-technical member. Sally is Vice President of
Marketing and Publicity for The Apache Software Foundation, and Chief Executive
of luxury brand communications consultancy HALO Worldwide.}

Everyone is a marketer. From the CEO to the superstar salesperson to the guy in
the mailroom, everyone is a representative of your company. Technologies and
tactics have changed over the years but good communications remain paramount. At
the end of the day, everyone is selling something, and it is an interesting
balance in publicity, as who and what you are and what you sell are often
enmeshed. When people tell me that they do not know who I am, I ask if they have
heard of W3C, Apache, or Creative Commons. The typical reply is ``of course'',
which assures me that I am doing my job. If you know who and what \textit{they} are,
things are good. It is about the product, not the publicist, after all. I never
set out to be in this space: cutting my communications teeth during the nascent
web years was not easy, but I am grateful to have had the opportunity to observe
others and dodge quite a few bullets. A sharp ramp-up and some very
highly-visible projects later, what advice would I share with a budding PR bunny,
seasoned media flack, or technologist daring to ride the promotions bucking bronco?

\section*{Never forget to declare yourself}
In selling your story to the press, remember that the media, too, have something
to sell. Sure, at the top level the role of a journalist is to tell a compelling
story (truthfully or not, factually or not, ethically or not, is another issue).
From attracting readership to securing subscriptions to promoting ad space, they
too are selling something, and your job is to help them do their job. The
reality is that some folks may not have heard of you, even if you have been around
for a long time. Or even if they have, they may not know who you are exactly. Be
clear with what it is that you have to offer. What is the press hook -- what is
the news? Be sure that the news is \textit{really} news. Be direct and get to the point
quickly. You have got to be prepared to answer the questions: ``So what?'' ``Why
should I care?'' ``What is in it for me?'', and that means having to ask questions
of yourself and your product. People buy ideas, not products, so promoting the
benefits of what you are pitching will help improve your chances of securing coverage.
Spin aside, what are you really selling?

\section*{Never on a Friday}
The worst day to launch a new website, issue a press release, or brief the media
is on a Friday. The chance that something wrong will happen with nobody
available to deal with the fallout is greater than you can imagine. A poignant
reminder of this happened to me early in my career when I launched the new W3C
homepage on a Friday evening, left the office and boarded a plane for Paris.
Coming from the world of commercial web publishing, using a proprietary tag
was not an issue whatsoever as long as it got the job done. Doing so on the
website of an interoperability-all-the-way organization on the other hand was
Not A Good Thing. Within minutes dozens of messages were pouring in, wondering
how the \textless now-deprecated-markup\textgreater -tag got on our site. And no, it was not \textless blink\textgreater \dots

\section*{Never think that it doesn't matter}
Credibility is everything. Despite being overworked, overcommitted or overextended,
you can not un-strike a bell. Try to deliver as much as you can to the best of
your ability and ask for help if you can. Some deadlines have to be adjusted,
and many editors can accommodate shift in schedule but it likely will not matter
(as much) once the story/fire has gone out if you are unable to follow through.
Like art, standards development, and copywriting, the process can go on ad
nauseam. Whilst creativity can not be time-managed, hard deadlines force a line to
be drawn at some point. But you have got to care about the details. Stop.
Proof-read and check all links. Make sure it maps properly to the overall
campaign/brand strategy. Lather-rinse-repeat is part of the greater
communications gestalt, and the work will keep piling up. Sort it out and
protect your reputation.


\section*{Do go at it alone}
It is important to trust your instincts, particularly when doing something
separate from the norm. In the early days of that newfangled web thaang,
everyone was seemingly tacking on the usual branding/PR/marketing tactics to a
brochure-ware Website. Then everyone was ``following the leader'' (leader = ``whoever did
it first'' in many instances). Trends are one thing, industry
expectations/requirements are another: ``that is how everybody does it'' does not
mean that it is right for you, your project or community. My career in
communications began when I fired our retained agency and brought everything
in-house. We were one of the earliest organizations to use a URL in a corporate
boilerplate, and were the first to use a URL as the originating location on a
press release dateline despite news wire agencies telling it was
non-conformant and against policy. Stand confidently in your knowledge. Go
against the grain and challenge the rules responsibly. Individuate. It is OK to be a
dissenter as long as you can back your ideas up.

\section*{Do provide perspective}
Many of the technologies I am involved with wind up in products 3-5 years down
the road. This means that, in many instances, it is hard to establish some sort
of relationship to a comparable product. It is critical that you explain your
position clearly with as little jargon as possible. Most non-developer
journalists/analysts I deal with do not follow the day-to-day activities of a
certain community or know the technical ins-and-outs of why one feature is
better than another, no matter how much of a no-brainer it is to you. The
saying of ``sell the sizzle, not the steak'' is more relevant today than ever.
Sizzle. Steak. There is always a split on this when I teach media training:
provide too much steak or too much sizzle and your campaign could fail.
Perception is key and the cause of a lot of conflict: All Sizzle = ``hype +
hyperbole'' = ``oh, you PR types''. All Steak = ``0s and 1s'' = ``oh you geek types''.
You need to understand and be able to clearly explain the painpoint that your product
solves. Knowing how to better present the problem allows you to better explain the
solution. Context, anecdotes, and success stories give the press a way to make their
readers care. You have got to know the answer to the question ``What is in it for
me?'', because that is what incents journalists to delve deeper into your story, which,
in turn, gets readers to learn more about you. Sizzle answers ``What’s in it for me?'',
and is therefore the hook. Steak is \textit{how} you get there.

\section*{Do queue up your spokespeople}
Always have someone available to talk to the press. Yes, it can be you, but know
that there will be a time that although you have a well-planned story to tell,
you may not be available to tell it. Who else do you work with? Who knows you?
Who endorses you? Defining those individuals and making a message map that
clarifies who says what helps alleviate an awful lot of potential headaches. I
usually act as the ``backgrounder'' spokesperson so I can spend time with a
reporter to find out what specifically are they looking for and how can we best
provide them with relevant information. I explain how things work, mostly
process-oriented; this puts my ``actual'' spokespeople in a better position to say
what they need, and minimize the risk of having their participation getting lost
elsewhere. Getting the right people ready is just as important as making them
available. In my media training classes, I include some ``Yikes!'' slides that
highlight particularly interesting lessons learned over the years. For example,
we experienced spokesperson mayhem in the early days of the Apache Incubator, where
15 people responded to a press query in 48 hours \dots\ lots of opinions, but who was the
``right'' one to quote? Do not leave it to the press to decide! Another oft-shared
``Yikes!'' scenario involved a global launch party with hundreds of guests, press
everywhere, DJs spinning, music blaring, cocktails flowing, and the event
running very late into the night, with rumored spin-off afterparties. Very early
the following morning the press queries came in (yes, of course I will accept a
phone call from the Financial Times at 4AM PT!). I pitched excitedly. However,
it turned out that we had no spokespeople available: Chairman on a plane to
Japan; Director's mobile phone was off (with reason, apparently); Board members
unavailable; staff unprepared. Dozens of opportunities missed. Remember: when
the press release goes out on the wire, the work has just begun.

\section*{Don't be surprised to take it from all sides}
Everyone has an opinion. And they will likely give it to you.

\section*{Don't overcomplicate things}
If you think you have got too much to tell, you probably do. Attention spans are
not what they were way back when; distraction/failure is just a click away.
Remember that you can always work in steps. Break up your story if needed. Cut a
lengthy press release and use supporting documentation such as technical fact
sheets and testimonial pages instead. The chunking principle (``5 plus or minus
two'') is something I continue to utilize again and again. Create your own
message release cycle, and reinforce your presence regularly. Bring a FAQ; if
there is a question that needs to be asked and is not there, find the opportunity
to bridge your message. Repetition breeds familiarity. Progressively reinforcing
your call to action is goodness.

\section*{Don't touch it for 24 hours}
Sometimes you need to walk away. From a project, from an argument, from work
altogether. Give yourself a break and try to pace yourself; allow a day for
things to settle down and for you to get a chance to breathe. Whilst that is
usually not possible in a deadline-driven industry, it is something to aim for.
The mad rush, non-stop emails, and continuous tweets often trigger reactions for
emergencies that do not exist. Put the project down, clear your head, and come
back with a fresh perspective. Step aside and regain your life.

\section*{Expect greatness}
Keep your standards high and know your worth.
