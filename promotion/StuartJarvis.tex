\chapter{Finding Your Feet in a Free Software Promotion Team - Stuart Jarvis}
\textit{Stuart Jarvis began working with the KDE Promotion Team in 2008 by
writing articles for KDE's news website, KDE.News. He learned the hard way how
to get things done in a free software community and got more involved with
promotion team activities such as writing KDE's release announcements and
getting articles about KDE software into the Linux press. He now sits on KDE's
Marketing Working Group, helping to set the direction of KDE's promotion and
marketing activities and helping new contributors to find their feet. He is also
now part of the editorial team for KDE.News, where his involvement with KDE
first began.}

``He who codes, decides'' is the mantra of free software development. But what
if there is no code? Or the he is a she?

Joining the promotion and marketing team of your favorite free software project
presents some special challenges. For new coders, most projects have code review
systems, maintainers and pre-releases of software that all help to spot errors
in code, making contributing your first patches less scary. 

Promotion can require your work to be visible to the public, with minimal
review, almost immediately. The non-hierarchical nature of free software
communities means there often is not a single person you can turn to who will
tell you whether your ideas are right and take some of the responsibility on
your behalf.

\section*{Getting consensus versus getting it done}

I first started contributing to KDE by writing articles for the official news
site, KDE.News. I had written for news outlets before, but always had a named
person to whom I would send a draft, receive feedback and then make changes as
required. In the KDE promotion team there was no single person or group of
people ``in charge''. I had to try and gauge the responses I got to draft
articles and decide whether I had all the feedback I needed and the article was
ready for publication.

With guidance from more experienced contributors, I eventually learned how to
propose something and get it published within a few days if there were no major
objections. The approach can be used by any contributor to a free Software
Promotion team, new or old alike.

First, work out how you would do something, whether it be writing an article,
changing a website text or giving a talk at your local school. Make a plan or
write the article or the new text. Send your ideas for review on the promotion
team mailing list of your organization. Importantly, do not ask people what they
think -- you can wait for days or weeks and not get definite answers. Instead,
state that you will publish or submit your text or execute your plan by a set
date in the future, pending any objections in the meantime.

When setting a deadline for comments, think about how long it will take everyone
active in the team to check email and consider your proposal. Twenty-four hours
is likely the absolute minimum for a simple yes or no answer to a
straightforward question. For something that requires reading or research, you
should allow several days.

If there are no big objections within the time limit you set, you can just go
ahead. If there are big problems with your plan, someone will tell you. Things
actually get done, you do not get frustrated with a lack of progress and you get
a reputation for completing tasks successfully.

\section*{Ultimately, it is your decision}

Free software communities can easily become discussion groups. Everyone has an
opinion. If you are not careful, discussions can become large, fade away as
people lose interest and finish without reaching any strong conclusions. That
can be hard enough to deal with when you have been around the community for a
while and have the experience to make your own decisions and your own views on
whose opinions you should listen to. When you are just starting out, it can be
very confusing.

If you want your own task to succeed, you may have to make decisions between
competing view points. You can wrap up the discussion by providing a summary of
the main points made and stating your opinion on them. Try not to leave any open
questions unless you want further discussion -- just state your conclusions and
what you are going to do. As long as you are reasonable, people are likely to
respect you even if they disagree.

\section*{Be proactive -- do not wait to be asked}

Your first contact with the promotion team you want to join may well be by
sending an email to their mailing list offering your skills. I thought I could
list things I was good at and expect people to suggest things for me to do.
Normally, it does not work quite like that.

Most communities are short of volunteers and really do need your skills.
However, because they lack volunteers, they can also lack time to provide good
guidance and mentoring. If there is a specific short-term project you would like
to work on, say so. It is much easier for someone in the project to simply say
``go ahead'' than to try and come up with a project to match your skills.

Even when you have worked on a few projects and proven your skills, you are
unlikely to often be approached personally with tasks. Those coordinating the
marketing team will not know your personal circumstances and so might not feel
comfortable asking you to do something specific in your own time, for free. An
ideal community will regularly post -- either on a mailing list or a web page --
tasks that volunteers can pick up. If that does not happen, find your own things
to do and tell the mailing list that you are doing them. People will notice and
it raises the chance that you will be directly approached in the future.

If you are proactive then you can quickly find that you are one of the
experienced people in the community that new people look to for advice and jobs
to work on. Try and remember what it was like when you started and make their
lives as new contributors as easy as possible.
