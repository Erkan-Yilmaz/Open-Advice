\chapter{Getting people together -- Dave Neary }

\textit{Dave Neary has been working on Free and Open Source projects since he
discovered Linux in 1996. A long-time contributor to GNOME and the GIMP,
he has worked full time helping companies come to terms with
community-developed software since 2007. In that time, he has worked
with projects including OpenWengo, Maemo and MeeGo on projects including
event organization, community processes, product management and
community metrics. As a volunteer, he has been involved in the
organisation of GUADEC, the Desktop Summit, the Libre Graphics Meeting,
the GIMP Conference, Ignite Lyon, the Open World Forum, and the MeeGo
Conference.
}

One of the most important things you can do in a free software project,
besides writing code, is to get your key contributors together as often
as possible.

I have been fortunate to be able to organize a number of events in the
past 10 years, and also to observe others and learn from them over that
time. Here are some of the lessons I have learned over the years from that
experience:

\section*{1. Venue}

The starting point for most meetings or conferences is the venue. If
you are getting a small group (under 10 people) together, then it is
usually OK just to pick a city, and ask a friend who runs a business or
is a college professor to book a room for you. Once you get bigger, you
may need to go through a more formal process.

If you are not careful, the venue will be a huge expense, and you will have
to find that money somewhere. But if you are smart, you can manage a
free venue quite easily.

Here are a few strategies you might want to try:
\begin{itemize}
 \item Piggy-back on another event -- the Linux Foundation Collaboration
Summit, OSCON, LinuxTag, GUADEC and many other conferences are happy to
host workshops or meet-ups for smaller groups. The GIMP Developers
Conference in 2004 was the first meet-up that I organized, and to avoid
the hassle of dealing with a venue, finding a time that suited everyone,
and so on, I asked the GNOME Foundation if they would not mind setting
aside some space for us at GUADEC -- and they said yes.
Take advantage of the bigger conference's organization, and you get the
added benefit of attending the bigger conference at the same time!
 \item Ask local universities for free rooms - This will not work once you go
over a certain size, but especially for universities which have
academics who are members of the local Linux User Group (LUG), they can talk their
department head into booking a lecture theatre and a few classrooms for a
weekend. Many universities will ask to do a press release and get credit
on the conference web-site, and this is a completely fair deal.
The first Libre Graphics Meeting was hosted free in CPE Lyon, and the
GNOME Boston Summit has been hosted free for a number of years in MIT.
 \item If the venue can not be free, see if you can get someone else to pay
for it. Once your conference is bigger than about 200 people, most
venues will require payment. Hosting a conference will cost them a lot,
and it is a big part of the business model of universities to host
conferences when the students are gone. But just because the university
or conference center will not host you for free that does not mean that you have
to be the one paying. Local regional governments like to be involved with big events in their
region. GUADEC in Stuttgart, the Gran Canaria Desktop Summit, and this
year's Desktop Summit in Berlin have all had the cost of the venue
covered by the host region. An additional benefit of partnering with the
region is that they will often have links to local industry and press --
resources you can use to get publicity and perhaps even sponsorship for
your conference.
 \item Run a bidding process -- by encouraging groups wishing to host the
conference to put in bids, you are also encouraging them to source a
venue and talk to local partners before you decide where to go. You are
also putting cities in competition with each other, and like olympic
bids, cities do not like to lose competitions they are in!
\end{itemize}

\section*{2. Budget}

Conferences cost money. Major costs for a small meet-up might be
covering the travel costs of attendees. For a larger conference, the
major costs will be equipment, staff and venue.

Every time I have been raising the budget for a conference, my rule of
thumb has been simple:
\begin{enumerate}
 \item Decide how much money you need to put on the event
 \item Fundraise until you reach that amount
 \item Stop fundraising, and move on to other things
\end{enumerate}

Raising money is a tricky thing to do. You can literally spend all of
your time doing it. At the end of the day, you have a conference to put
on, and the amount of money in the budget is not the major concern of
your attendees.
Remember, your primary goal is to get project participants together to
advance the project. So getting the word out to prospective attendees,
organizing accommodation, venue, talks, food and drinks, social
activities and everything else people expect at an event is more
important than raising money.
Of course, you need money to be able to do all the rest of that stuff,
so finding sponsors, fixing sponsorship levels, and selling your
conference is a necessary evil. But once you have reached the amount of
money you need for the conference, you really do have better things to
do with your time.

There are a few potential sources of funds to put on a conference -- I
recommend a mix of all of these as the best way to raise your budget.

\begin{itemize}
 \item Attendees -- While this is a controversial topic among many
communities, I think it is completely valid to ask attendees to
contribute something to the costs of the conference. Attendees benefit
from the facilities, the social events, and gain value from the conference.
Some communities consider attendance at their annual event as a kind of
reward for services rendered, or an incitement to do good work in the
coming year, but I do not think that's a healthy way to look at it.
There are a few ways for conference attendees to fund the running of
the conference:
 \begin{enumerate}
   \item Registration fees -- This is the most common way to get money from
conference attendees. Most community conferences ask for a token amount
of fees. I have seen conferences ask for an entrance fee of 20 to 50 Euro,
and most people have not had a problem paying this.
A pre-paid fee also has an additional benefit of massively reducing
no-shows among locals. People place more value on attending an event
that costs them 10 Euro than one where they can get in for free, even if the
content is the same.
   \item Donations -- very successfully employed by FOSDEM. Attendees are
offered an array of goodies, provided by sponsors (books, magazine
subscriptions, t-shirts) in return for a donation. But those who want
can attend for free.
   \item Selling merchandising -- Perhaps your community would be happier
hosting a free conference, and selling plush toys, t-shirts, hoodies,
mugs and other merchandising to make some money. Beware: in my
experience you can expect less from profits from merchandising sales
than you would get giving a free t-shirt to each attendee with a
registration fee.
 \end{enumerate}
 \item Sponsors -- Media publications will typically agree to ``press
sponsorship'' -- providing free ads for your conference in their print
magazine or website. If your conference is a registered non-profit which
can accept tax-deductible donations, offer press sponsors the chance to
invoice you for the services and then make a separate sponsorship grant
to cover the bill. The end result for you is identical, but it will
allow the publication to write off the space they donate to you for tax.
What you really want, though, are cash sponsorships. As the number of
free software projects and conferences has multiplied in recent years,
the competition for sponsorship dollars has really heated up in recent
years. To maximize your chances of making your budget target, there are
a few things you can do.
 \begin{enumerate}
  \item Conference brochure -- Think of your conference as a product you are
selling. What does it stand for, how much attention does it get, how
important is it to you, to your members, to the industry and beyond?
What is the value proposition for the sponsor?
You can sell a sponsorship package on three or four different grounds:
perhaps conference attendees are a high-value target audience for the
sponsor, perhaps (especially for smaller conferences) the attendees
are not what is important, it is the attention that the conference will get
in the international press, or perhaps you are pitching to the company
that the conference is improving a piece of software that they depend on.
Depending on the positioning of the conference, you can then make a
list of potential sponsors. You should have a sponsorship brochure that
you can send them, which will contain a description of the conference, a
sales pitch explaining why it is interesting for the company to sponsor
it, potentially press clippings or quotes from past attendees saying how
great the conference is, and finally the amount of money you are looking for.
  \item Sponsorship levels -- These should be fixed based on the amount of
money you want to raise. You should figure on your biggest sponsor
providing somewhere between 30\% and 40\% of your total conference budget
for a smaller conference. If you are lucky, and your conference gets a
lot of sponsors, that might be as low as 20\%. Figure on a third as a
ball-park figure. That means if you have decided that you need 60,000 Euro
then you should set your cornerstone sponsor level at 20,000 Euro, and all
the other levels in consequence (say, 12,000 Euro for the second level and
6,000 Euro for third level).
For smaller conferences and meet-ups, the fundraising process might be
slightly more informal, but you should still think of the entire process
as a sales pitch.
  \item Calendar -- Most companies have either a yearly or half-yearly
budget cycle. If you get your submission to the right person at the
right time, then you could potentially have a much easier conversation.
The best time to submit proposals for sponsorship of a conference in the
Summer is around October or November of the year before, when companies
are finalizing their annual budget.
If you miss this window, all is not lost, but any sponsorship you get
will be coming out of discretionary budgets, which tend to get spread
quite thin, and are guarded preciously by their owners. Alternatively,
you might get a commitment to sponsor your July conference in May, at
the end of the first half budget process - which is quite late in the day.
  \item Approaching the right people -- I am not going to teach anyone sales,
but my personal secret to dealing with big organizations is to make
friends with people inside the organizations, and try to get a feel for
where the budget might come from for my event. Your friend will probably
not be the person controlling the budget, but getting him or her on
board is your opportunity to have an advocate inside the organization,
working to put your proposal in front of the eyes of the person who owns
the budget.
Big organizations can be a hard nut to crack, but free software
projects often have friends in high places. If you have seen the CTO or
CEO of a Fortune 500 company talk about your project in a news article,
do not hesitate to drop him a line mentioning that, and when the time
comes to fund that conference, a personal note asking who the best
person to talk to will work wonders. Remember, your goal is not to sell
to your personal contact, it is to turn her into an advocate to your
cause inside the organization, and create the opportunity to sell the
conference to the budget owner later.
 \end{enumerate}
 \item Also, remember when you are selling sponsorship packages that everything
which costs you money could potentially be part of a sponsorship
package. Some companies will offer lanyards for attendees, or offer to
pay for a coffee break, or ice-cream in the afternoon, or a social
event. These are potentially valuable sponsorship opportunities and you
should be clear in your brochure about everything that is happening, and
spec out a provisional budget for each of these events when you are
drafting your budget.
\end{itemize}

\section*{3. Content}

Conference content is the most important thing about a conference.
Different events handle content differently -- some events invite a large
proportion of their speakers, while others like GUADEC and OSCON invite
proposals and choose talks to fill the spots.

The strategy you choose will depend largely on the nature of the event.
If it is an event in its 10th year with an ever increasing number of
attendees, then a call for papers is great. If you are in your first
year, and people really do not know what to make of the event, then
setting the tone by inviting a number of speakers will do a great job of
helping people know what you are aiming for.

For Ignite Lyon last year, I invited about 40\% of the speakers for the
first night (and often had to hassle them to put in a submission, and
the remaining 60\% came through a submission form. For the first Libre
Graphics Meeting, apart from lightning talks, I think that I contacted
every speaker except 2 first. Now that the event is in its 6th year,
there is a call for proposals process which works quite well.

\section*{4. Schedule}

Avoiding putting talks in parallel which will appeal to the same people
is hard. Every single conference, you hear from people who wanted to
attend talks which were on at the same time on similar topics.

My solution to conference scheduling is very low-tech, but works for me.
Colored post-its, with a different color for each theme, and an empty
talks grid, do the job fine. Write the talk titles one per post-it, add
any constraints you have for the speaker, and then fill in the grid.

Taking scheduling off the computer and into real life makes it really
easy to see when you have clashes, to swap talks as often as you like,
and then to commit it to a web page when you are happy with it.

I used this technique successfully for GUADEC 2006\footnote{\url{http://blogs.gnome.org/bolsh/2006/05/09/initial-schedule-ready}} and Ross
Burton re-used it successfully in 2007\footnote{\url{http://www.flickr.com/photos/rossburton/467140094}}.

\section*{5. Parties}

Parties are a trade-off. You want everyone to have fun, and hanging out
is a huge part of attending a conference. But morning attendance suffers
after a party. Pity the poor community member who has to drag himself
out of bed after 3 hours sleep to go and talk to 4 people at 9am after
the party.

Some conferences have too many parties. It is great to have the
opportunity to get drunk with friends every night. But it is not great to
\textit{actually} get drunk with friends every night. Remember the goal of the
conference: you want to encourage the advancement of your project.

I encourage one biggish party, and one other smallish party, over the
course of the week. Outside of that, people will still get together, and
have a good time, but it will be on their dime, and that will keep
everyone reasonable.

With a little imagination, you can come up with events that do not
involved loud music and alcohol. Other types of social event can work
just as well, and be even more fun.

At GUADEC we have had a football tournament for the last number of
years. During the OpenWengo Summit in 2007, we brought people on a boat
ride on the Seine and we went on a classic 19th century merry-go-round
afterwards. Getting people eating together is another great way to
create closer ties. I have very fond memories of group dinners at a
number of conferences. At the annual KDE conference Akademy, there is
typically a Big Day Out, where people get together for a picnic, some
light outdoors activity, a boat ride, some sightseeing or something similar.

\section*{6. Extra costs}

Watch out for those unforeseen costs! One conference I was involved in,
where the venue was ``100\% sponsored'' left us with a 20,000 Euro bill for
labor and equipment costs. Yes, the venue had been sponsored, but
setting up tables and chairs, and equipment rental of whiteboards,
overhead projectors and so on, had not. At the end of the day, I
estimate that we used about 60\% of the equipment we paid for.

Conference venues are hugely expensive for everything they provide.
Coffee breaks can cost up to 10 US Dollar per person for a coffee and a few
biscuits, bottled water for speakers costs 5 US Dollar per bottle, and so on.
Rental of an overhead projector and mics for one room for one day can
cost 300 Euro or more, depending on whether the venue insists that equipment
be operated by their a/v guy or not.

When you are dealing with a commercial venue, be clear up-front about
what you're paying for.

\section*{7. On-site details}

I like conferences that take care of the little details. As a speaker, I
like it when someone contacts me before the conference and says they will
be presenting me, what would I like them to say? It is reassuring to know
that when I arrive there will be a hands-free mic and someone who can
help fit it.

Taking care of all of these details needs a gaggle of volunteers, and it
needs someone organizing them beforehand and during the event. Spend a
lot of time talking to the local staff, especially the audio/visual
engineers.

In one conference, the a/v guy would switch manually to a screen-saver
at the end of a presentation. We had a comical situation during a
lightning talk session where after the first speaker, I switched
presentations, and while the next presentation showed up on my laptop,
we still had the screensaver on the big screen. No-one had talked to the
a/v engineer to explain to him the format of the presentation!
So we ended up with 4 Linux engineers looking at the laptop, checking
connections and running various Xrandr incantations, trying to get the
overhead projector working again! We eventually changed laptops, and the
a/v engineer realized what the session was, and all went well after that
-- most of the people involved ended up blaming my laptop.

Running a conference, or even a smaller meet-up, is time consuming, and
consists of a lot of detail work, much of which will never be noticed by
attendees. I have not even dealt with things like banners and posters,
graphic design, dealing with the press, or any of the other joys that
come from organizing a conference.

The end result is massively rewarding, though. A study I did last year
of the GNOME project showed that there is a massive project-wide boost
in productivity just after our annual conference, and many of our
community members cite the conference as the high point of their year.

\todo{write closing words}
