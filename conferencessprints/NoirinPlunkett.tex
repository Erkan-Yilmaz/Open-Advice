\chapter{People are Everything - Nóirín Plunkett}

\textit{Nóirín Plunkett is a jack of all trades, and a master of several. A technical writer by day, her Open Source work epitomizes the saying ``if you want something done, ask a busy person''.
Nóirín got her Open Source start at Apache, helping out with the httpd documentation project. Within a year, she had been recruited to the conference planning team, which she now leads. She was involved in setting up the Community Development project at Apache, has previously acted as Org Admin for the Google Summer of Code (with more than 40 students!), and continues to contribute to projects as diverse as Infrastructure and Incubator. She sits on the boards of both the Apache Software Foundation and the Open Cloud Initiative.
At home in Ireland, Nóirín was a volunteer with the St John Ambulance -- since moving to Switzerland, she’s had to find new ways to help save the world. Happily, Open Source has opened more than just technical doors, and when Christchurch suffered a devastating earthquake earlier this year, Nóirín’s knowledge of OS disaster management software meant she could quickly step up to co-ordinate the night shift of volunteers working on the Christchurch Recovery Map at http://eq.org.nz.
When she’s not online, Nóirín’s natural habitat is the dance floor, although she’s also a keen harpist and singer, and an excellent sous chef!}\todo{needs shortening}

There is no such thing as a typical path, although mine was perhaps
less typical than most. I first got a commit bit in my twenties, by
which time I had already spent more than a year working at Microsoft.
But after Microsoft I had moved to a foreign country to continue my
studies, and it was nice to have a distraction, so I started working
on various docs and translations, and I got a commit bit on the Apache
httpd project.

As luck would have it, of course, ApacheCon EU was going to be held in
Dublin the summer I was studying in Munich. But luck is kind to the
Irish, and with only a little bit of wangling, I persuaded Sun
Microsystems to sponsor me to attend the conference.

They say ``show, don't tell'', so let me show you where I was, the
moment I realized that this Open Source thing was for real, was going
to change the world:
http://www.flickr.com/photos/rous/175453160/ \todo{How do we do this in the book?}

It was the evening before the conference. We still had not figured out
where the fibre was terminated, that was supposed to make up our
network backbone. We had checked every corner, cupboard and skirting, to
no avail. We had given up for the night, and were busy trying to make
sure that the rooms that would be hosting training classes the next
day had at least enough connectivity for the trainers to demonstrate
their material\footnote{The next morning, we checked up in the roof space, to try and find
the fibre; still no joy. In the end, we found it in the comms cupboard
of the nightclub in the basement next door.}.

And as evening turned into night, and routers slowly revealed their
Default Configuration secrets, half a dozen volunteers, people I had
only met that afternoon, became friends.

I could not tell you where the half dozen girls I lived with that
summer in Munich are now. But I am still in contact with each of the
people you see in that picture. One of them has moved to a different
country, another to a different continent. Most of them have changed
jobs in the meantime, and I have graduated, taking up the grand Irish
tradition of emigration to find employment.

You see, Open Source is all about the people. Really, on almost any
project you woud want to be a part of, the code comes second. People are
what distinguish a project that is a joy to work on from one that is a
chore; people are what make the difference between a project that is
flourishing and one that languishes in the bitbucket. Sure, you will
only stay up all night coding on a project if it is solving a problem
you think is important; but unless you have people with whom you can
collaborate, discuss, design, and develop, you are probably going to
lose interest or get stuck before too long.

The true value of conferences, sprints, hackathons, retreats, or
whatever your community calls their face-to-face moments, is exactly
that. Coming face-to-face with the people you have been working with.
Human beings are social animals; babies recognize faces even before
they begin babbling, and no matter how good people are about being
friendly and polite in email, there is something lost in those
communications.

Meeting people face to face gives us an opportunity to recognize the
humanity in those we might have struggled to get along with; to share
the joy of a job well done with those we love to work with. Therefore,
if I could have chosen one piece of advice, to hear when I was
starting out, it would be to get out there, to meet people, to put
faces to names at every opportunity\footnote{Sadly, I do say this with a caveat; as with any large gathering of
people, there are risks to attending an Open Source conference. Some
are worse than others, but in my own experience, assault in particular
seems to be more prevalent in technical communities than in the
non-technical. Seek out events that have a published code-of-conduct
or anti-harassment policy, and ask for backup if you feel unsafe. The
vast majority of the people you will find at an Open Source event are
wonderful, caring human beings; I hope that in time, changing
attitudes will stop the minority from thinking that they can get away
with unreasonable behavior in these venues.}

And if you find the opportunities are few and far between, do not be
afraid to ask. Look for people who are traveling near you, or who
live where you are traveling; seek sponsorship to attend the larger
community events; organize an event of your own!

It is the richness of our communities that makes Open Source what it
is, and the shared striving towards common goals. And of course, the
music sessions, the meals, the pints, and the parties! These are the
things that bring us together, and you will find that once you have met
people in person, even your email interactions will be much richer,
much more fulfilling, and much more fruitful, than they had previously
been.
