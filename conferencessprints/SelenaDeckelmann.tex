\authorbio{Selena Deckelmann is a major contributor to PostgreSQL. She speaks
internationally about free software, developer communities and trolling. Her
interests include opening up government data with the City of Portland, urban
chickens and finding ways to make databases run faster.
\newline
She founded Postgres Open, a conference
dedicated to the business of PostgreSQL and disruption of the database industry.
She founded and co-chaired Open Source Bridge, a developer conference
for open source citizens. She founded the PostgreSQL Conference, a successful
series of east coast/west coast conferences in the US for PostgreSQL. She is
currently on the program committees of PgCon and MySQL Users Conference, and
OSCON data. She's a contributing writer for the Google Summer of Code Mentor
Manual, and Student Guide. She is an advisor to the Ada Initiative and board
member of Technocation, Inc.}

\chapter{Selena Deckelmann}{How to Ask for Money}

Looking back since the first time I booted a PC into Linux in 1994, one thing
stands out in my experience with open source: I wish I had known how to
ask for money.
Asking for money is hard. I have written grant proposals, asked for raises,
negotiated salaries and consulting hourly rates, and raised funds for non-profit
conferences. Through much trial and error, I have developed a process that
works!
What follows is a distillation of the tricks and techniques I have used over the
last five years to raise money for unconferences, day-long code sprints and
multi-day conferences about open source software and culture.

The process of getting money for a conference is really about six steps: 
\begin{enumerate}
 \item Identify a need. 
 \item Tell someone. 
 \item Ask for money.
 \item Get the money.
 \item Spend the money. 
 \item Say thank you.
\end{enumerate}

\section*{Identify a need}

Your first task as a conference organizer is to explain why you are putting on
yet another conference, why that conference will be useful to attendees and why
a sponsor should give you money to do it. This is called ``writing a
prospectus.''
The main elements of a prospectus are: 

\begin{itemize}
\item Purpose:
In a paragraph, explain why you are having the conference. What inspired you to
bring people together? And who are the attendees? What will they talk about once
they are there? 

If you have got a theme, or a specific goal in mind, mention that. Also, explain
why you picked the location for the event. Is there some tie to the theme of the
conference? Are the right people in that location? Was it sponsored by someone?

Finally, share any interesting numbers from previous events, like number of
attendees, interesting facts about speakers or details about your chosen
location. 

\item Sponsorship opportunities and benefits:
This section of the prospectus will outline what sponsors can expect from your
conference. Typically, this is organized by dollar amount, but could also
describe benefits for in-kind or volunteer work.

Start simple. Typically, sponsorships for events with cash are arranged by HR
departments looking to hire, or marketing departments looking to advertise
products or services. 

The types of benefits sponsors ask for include: recognition on a website,
mention of sponsorship in email or tweets out to attendees, access to email
addresses and/or demographic information about attendees, logo and labels on
conference totebags, lanyards or other swag, coffee breaks and lunch, parties,
conference booth space and advertising space in a conference program. 

Also, consider creative things that are unique to you, the conference and the
location. For example, Portland has a very popular doughnut shop with a truck
delivery service. We got a sponsor and then acquired permission to drive the
truck right onto the grounds of our venue and served free doughnuts for
breakfast.

Links to example prospectuses are below. They are all for big conferences, so
YMMV. I have made a prospectus before that
only had one option for sponsorship, and the benefits were: send one attendee
from your company, and the organizers will publicly recognize your company and 
thank you for your sponsorship.
\begin{itemize}
\item OSCON: \url{http://bit.ly/zd62Q6}
\item Open Source Bridge: \url{http://bit.ly/woHCxf}
\item MeeGo San Francisco: \url{http://bit.ly/zLUKEN}
\end{itemize}

\item Contract:
Always include a contract with your prospectus. This establishes basic
expectations and timelines, and can save you a lot of trouble down the road.

I am not a lawyer, and so what follows is my experience rather than legal
advice. For smaller events, I write a very simple contract that outlines my
expectations: sponsors promise to pay by a certain date, and I promise to hold
the event on a certain date.

Copying an existing contract is a tricky business, as laws change and vary
across states and countries. I consulted a lawyer that was recommended to me by
an experienced open source community manager. The law firm was nice enough to
create contracts and review contracts with hotels with us on a pro-bono basis.
The Software Freedom Law Center may be able to refer you to an appropriate
lawyer if you do not have one.
\end{itemize}

Now that you have created the prospectus, you need to talk to some people.

\section*{Tell someone}
The most difficult step for me personally is getting the word out about my
events! 

Practice explaining your event in 1-2 sentences. Distill out what excites you,
and what should excite other people.

Over the years, I have learned that I need to start talking RIGHT NOW to the
people that I know, rather than worrying a whole lot about exactly the right
people to tell.  Make a list of people to talk to that you know already, and
start checking them off.

The best way to start talking about what you are doing is in person or on the
phone. This way, you are not spamming people, you have their attention, and you
can get immediate feedback about your pitch. Do people get excited? Do they ask
questions? Or do they get bored? Who else do they think you should talk to? Ask
for feedback, and how you can make your pitch more appealing, interesting and
worth their money!

Once you have your verbal pitch down, write it up and send a few emails. Ask for
feedback on your email and always close the email with a call to action and a
timeline for response.  Keep track of who responds, how they respond and when
you should follow up with each person.

\section*{Ask for money}
Armed with your prospectus, and your finely tuned pitch, start approaching
companies to fund your event. Whenever I start a new conference, I make a list
of questions about my conference and answer each with a list of people and
companies: 
\begin{itemize}
\item Which people do I know who will think this is an amazing idea and will
advocate for my event? (Cheerleaders)
\item Who would be really fun to have around at the conference? (Mavens)
\item Which companies have products that they want to pitch at my event?
(Marketing)
\item Who would want to hire the people who attend? (Recruiters)
\item Which free and open source projects would like to recruit developers?
(Open Source Recruiters)
\end{itemize}

Using these lists, send your prospectus out into the world! Here is an overview
of how I organize the asking process: 
I start by sending prospectuses to my Cheerleaders. I also drop a copy of the
prospectus with the Mavens, and invite them to attend the conference or speak. I
then contact Marketing companies, Recruiters and Open Source Recruiters
(sometimes there is overlap!).
Meanwhile, I typically have opened registration for the conference and announced
a few keynotes or special events. Hopefully this drives registrations a bit, and
helps make sponsors feel like the conference is definitely going to happen, and
that things are going well.

\section*{Get the money}
If everything goes according to plan, companies and people start offering you
money. When this happens you need two very important things: 
\begin{itemize}
\item An invoice template
\item A bank account to hold the money
\end{itemize}

Invoice templates are simple. I have a Google Spreadsheet that I just update for
each invoice. You could easily use Open Office or even TeX (please, someone send
me a LaTeX invoice template!) Examples of what invoices look like are available
at \url{http://www.freetemplatesdepot.com}.

The most important elements of invoices are: the word INVOICE, a number for the
invoice that is unique, the name and contact information of the sponsor, what the
sponsor is expected to pay, terms of the invoice (when the sponsor should pay
by, and what the penalty is for non-payment) and the total amount due. Then you
need to send a copy of this form to the company. Keep a copy for yourself!

Some companies may require simple or complicated forms to be filled out and
signed to identify you or your organization as a vendor. Paperwork. Ugh! Payment
cycles for large companies can be up to two months. Also, budget cycles for
companies are typically yearly. Find out whether a company even \textit{has}
available budget for your event, and whether you can get into their budget the
following year if you missed the current year’s window.

The bank account can be your personal bank account, but this puts you at risk.
For a many-thousand-dollar event, you may wish to find an NGO or non-profit
organization that can hold and dispense funds for you. If your conference is
for-profit, you should consult an accountant about how to organize the funds.
Finding a non-profit to work with may be as simple as contacting a foundation
associated with an open source project. 

Now on to what makes this whole process worthwhile - spending your hard-earned
sponsorships!

\section*{Spend money}
Now that your sponsors have paid, you can spend the money. 

Create a budget that details what you want to spend money on, and when you will
need to spend it. I recommend getting 3 quotes for products and services you are
unfamiliar with, just so you can get a sense of what a fair price is. Let
companies you are contacting know that you are going through a competitive bid
process. 

Once I establish a relationship with a company, I tend to do business with them
year after year. I like having relationships with vendors, and find that even if
I pay slightly more than if I aggressively bid things out every year, I end up
saving time and getting better service from a vendor that knows me well. 

For small events, you can keep track of expenses in a fairly simple spreadsheet.
For larger projects, asking an accountant, or using dedicated accounting software
can help. Here is a list of Quicken alternatives that are free (to varying degrees
and in varying aspects!): \url{http://bit.ly/9RRgu0}

What is most important is to keep track of all your expenses, and to not spend
money that you do not have! If you are working with a non-profit to manage the
event’s money, ask them for help and advice before getting started.

\section*{Say thank you}
There are many ways to say thank you to the people and companies that supported
your event. Most importantly, follow up on all the promises you made in the
prospectus. Communicate as each commitment is met!

During the event, find ways of connecting with the sponsors, by designating a
volunteer to check in with them and checking in with them yourself.

After the event, be sure to individually thank sponsors and volunteers for their
contributions. A non-profit I work with sends thank-you notes individually to
each sponsor at the start of the new year.

Generally speaking, communication is the compost of fundraising! Giving
attention and building genuine relationships with sponsors helps find more
sponsors, and build your reputation as a great event organizer.

\section*{Lessions learned}
After creating and running dozens of events, the two most important aspects of
it all have been finding mentors and learning to communicate well. 

Mentors helped me turn rants into essays, messes into prospectuses, and
difficult conversations into opportunities. I found mentors at companies that
sponsored my conferences and gave detailed, sometimes painful, feedback. And I
found mentors among volunteers who dedicated hundreds of hours to write software
for my events, recruit speakers, document what we were doing, and carry the
conference on after me. 

Learning to communicate well takes time, and the opportunity to make a lot of
mistakes. I learned the hard way that not developing a relationship with the
best sponsors means no sponsorship the following year! I also found that people
are incredibly forgiving when mistakes happen, as long as you communicate early
and often.

Good luck with your fundraising, and please let me know if you find this
helpful.
