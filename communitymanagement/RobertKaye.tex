\authorbio{Robert Kaye combines his love for music and open source into the open
music encyclopedia MusicBrainz. Robert founded and leads the California-based
non-profit MetaBrainz Foundation in a long term effort to improve the digital
music experience. Beyond hacking on MusicBrainz, Robert seeks out interesting
festivals like Burning Man and interesting side projects like hacking on
drink-mixing robots. Topped with a colorful hair style at all times, you will
never have a hard time picking him out of a crowd.}

\chapterwithauthor{Robert Kaye}{What I Wish I Would Have Known When I Started the CD Index...}

\todo{title}

In 1998, I was working at Xing Technology in San Luis Obispo, working hard on
our new AudioCatalyst project. It was one of the first integrated MP3 ripping
programs that made use of the CDDB database. CDDB was the CD database that
allows any player to look up the title and tracklisting for any CD. If the CD
was not listed, you could enter the data so that the next person could make us
of the data. I loved this online collaborative project and typed in several
hundred CDs over the course of a few years.

One day we were notified that CDDB had been purchased by Escient, a company that
would later become GraceNote. The CDDB database was taken private so that people
could no longer download the complete database! And on top of that Escient did
not compensate any of the contributors for their efforts; they were ripping off
the general public with this move. I was quite angry with this move and still am
to this day.

A few months later we were notified by Escient that we would be required to play
the Escient jingle and display the Escient logo when making a CD lookup in our
products. That was it! Now I was livid! Later that week at a party with friends
I was complaining about what was happening and how unhappy I was. My friend
Kevin Murphy said to me: ``Why don’t you start your own open source project to
compete with these bastards?''

A few weeks later I stopped working for Xing and had a couple of weeks of spare
time before I would start at EMusic. I decided to teach myself Perl and web
programming and set out to create the CD Index, a non-compatible, non-infringing
project to compete with CDDB. I hacked on the project during the break, but then
promptly forgot it once I became a member of the FreeAmp project at EMusic.

Then in March of 1999 Slashdot asked what the open replacement for CDDB was
going to be. I spent the rest of that day and most of the night finishing the CD
Index and deploying it. I submitted a Slashdot story about my
project\footnote{\url{
http://slashdot.org/story/99/03/09/0923213/OpenSource-Alternative-to-CDDB}} and
it promptly posted. As expected, thousands of geeks showed up within minutes and
my server tipped over and died.

The masses of people who arrived immediately started shouting for things to
happen. There was not even a mailing list or a bug tracker yet; they insisted on
having one right now. Because I was new to open source, I did not really know
what all was needed to launch an open source project, so I just did as people
asked. The shouting got louder and more people insisted that I shut the service
down because it was not perfect. Even amidst the mess, we received over 3000 CD
submissions during the first 24 hours.

Once things calmed down, there were still plenty of people shouting. Greg Stein
proclaimed that he would write a better version immediately. Mike Oliphant,
author of Grip, said he was going to work on a new version as well. Alan Cox
came and loudly proclaimed that SQL databases would never scale and that I
should use DNS to create a better CD lookup service. Wait, what?
I was very unhappy with the community that grew out of the Slashdot posting. I
did not want a place were people could treat each other without respect and
people who felt entitled could shout louder until they got what they wanted. I
quickly lost interest in the project and the CD Index faltered. The other
projects that people promised they would start (not counting FreeDB) never
materialized.

Then when the dot com bust started, I needed to think about what I would do
next. It was clear that my job at EMusic was not safe; still I was driving a
Honda S2000 roadster, my dot com trophy car. With car payments double my rent, I
had to decide: Work on my own stuff and sell my dream car, or move to the Bay
Area and work on someone else’s dream, if I could even find a job there.

I decided that a comprehensive music encyclopedia that was user-generated would
be the most interesting thing to work on. I sold the S2000 and hunkered down to
start working on a new generation of the CD Index. At yet another party, the
name MusicBrainz came to me and I registered the domain in the middle of the
party. The next day, motivated by the project’s new name, I started hacking in
earnest and in the Fall of 2000 I launched musicbrainz.org.

Launched is not the right term here -- I set up the site quietly and then
wondered how I could avoid another Slashdot-based community of loud screaming
kids. I never imported data from the CD Index, nor did I mention MusicBrainz on
the CD Index mailing lists. I simply walked away from the CD Index; I wanted
nothing more to do with it. In the end I decided to add one simple button to the
FreeAmp web page that mentioned MusicBrainz.

And a very strange thing happened: people came and checked out the project. It
was very few people at first, but when a person mentioned something to me, I
would start a conversation and gather as much feedback as I could. I would
improve the software based on feedback. I also set a tone of respect on the
mailing lists, and every time someone was disrespectful, I would step in and
speak up. My efforts directed the focus of the project towards improving the
project. I did this for over 3 years before it became clear that this approach
was working. The database was growing steadily and the data quality went from
abhorrent to good over a number of years. Volunteers come and go, but I am the
constant for the project, always setting the tone and direction for the project.
Today we have a 501(c)3 non-profit with 3.25 employees in 4 countries, Google,
the BBC and Amazon as our customers and we are in the black. I doubt that could
have happened with the CD Index community.

I wish I would have known that communities need to grow over time and be
nurtured with a lot of care.
