\chapter{Hindsight is Almost 20/20 - Jono Bacon}

\textit{Jono is a community manager, engineering manager, consultant and author.
Currently he works as the Ubuntu Community Manager at Canonical, leading a team
to grow, inspire and enthuse the global Ubuntu community. He is the author of
Art of Community, founder of the Community Leadership Summit and co-founder of
the popular podcast LugRadio.}

I first learned of Linux and Open Source back in 1998. While the technology was
gnarly and the effort required to get a smooth running system was significant,
the concept of this global collaborative community transfixed me. Back then I
had no knowledge, limited technical skills, and zits.

As an angsty teenager complete with long hair and Iron Maiden t-shirt, my path
was really already mapped out for me in the most traditional sense; I would go
to school, then college, then university, and then a job.

Fourteen years later, the path I actually took was by no means traditional, and
that intrinsic fascination with community has taken me around the world and
thrown me into some engrossing challenges. It is interesting to sit back and
reflect on this period of time. Well, it might be interesting for me \dots you
might want to skip to the next chapter \dots 
\newline
\dots
\newline
Still with me? OK, let’s roll.

\section*{Science vs. Art}

I have always believed that community management is less of a science and more
of an art. I define science as exploring methods of reproducing phenomena
through clearly understood and definitive steps. In the science world if you
know the theory and recipe for an outcome, you can often reproduce that outcome
like anyone else.

Art is different. There is no recipe for producing an incredible song, for
creating an amazing painting, or sculpting a beautiful statue. Similarly, there
is not really any reproducible set of steps for creating a thriving community.
Sure, there are tricks and techniques for achieving components of success, but
the same happens for other art-forms; we can all learn the notes and chords on a
guitar, it does not mean you are going to write the next Bohemian Rhapsody. The
formula that generates Bohemian Rhapsody is one part learned skill and one part
magic.

Now, I am not suggesting that community management is this frustratingly hip and
introverted artform that only the blessed few with such talents can achieve.
What I am instead lamenting is that there is no playbook for how to create a
wonderful and inspiring community; it is still one part learned skill and one
part magic, but the magic part is not divinely anointed to you by the gods, but
instead obtained by trying new things, being receptive to feedback, and getting
a feel for what works and what does not.

Rather frustratingly, this means that there is no single recipe to follow for
the magic, but there is still an opportunity to share the learned skills, as I
have sought to do with The Art of
Community\footnote{\url{http://artofcommunityonline.org}} and the annual
Community Leadership
Summit\footnote{\url{http://communityleadershipsummit.com}}.

Before I get started reflecting, and for those of you who have not bored
yourself into oblivion by following my career, I will summarize the communities
I have worked with so we can define the context. In a nutshell, I started out in
my hairier days by producing one of the UK’s first Linux community websites
called Linux UK and got involved in the Linux User Group (LUG) community. I went
on to create my own LUG in Wolverhampton in the UK and founded the Infopoint
project to encourage LUGs to advocate Linux at computer fairs across the UK. I
then went on to contribute to the KDE community, founded the KDE::Enterprise
site, got the KDE Usability Study going, and contributed to a few little apps
here and there. I then founded the PHP West Midlands user group and started also
getting interested in GNOME. I wrote a few apps (GNOME iRiver, XAMPP Control
Panel, Lernid, Acire) and also co-designed and wrote some code for a new
simplified audio app called Jokosher. Around this time I co-founded the LugRadio
podcast which would run for four years with over two million downloads and
spawning five live events in the UK and USA. At this time I also started work as
an Open Source consultant at the government-funded OpenAdvantage where I really
got a chance to cut my teeth in community and working with organizations across
the West Midlands to help them to move to Open Source. After a few years at
OpenAdvantage I moved on to join Canonical as the Ubuntu community manager and
developed a team of four and together we are involved in a wide variety of
projects in Ubuntu and Canonical.
\newline
Still with me?
\newline
Wow, you are persistent. Or bored. Probably bored. There will be an exam at the
end; that’ll learn you \dots

\section*{Reflecting}

So this brings me to the focus of this piece -- the curious question of if I knew what I did today, what would I tell myself? Over the course of my career so far I believe that everything I have learned can be boiled into two broad buckets:
\begin{itemize}
 \item Practical -- the tips and tricks of the trade; e.g. approaches to
communication mediums, using technology in different ways, event planning
techniques, project management approaches etc.
 \item Personal -- core life lessons and learnings that affect the approach you
take to your world.
\end{itemize}
I am not going to talk much about the practical -- you should read my
book for more on that topic (the book also covers a lot of the personal too).
Today I am instead going to focus on the personal life lessons. Approaches and
practices will always change, but the life lessons do not so much change but
grow and evolve as we get wiser.

\section*{The Importance Of Belief}

Communities are fundamentally networks of people driven by belief. Every
community has an ethos and a focus. This could be something as grandiose as
documenting all human knowledge or changing the world with Free Software, or it
could be as humble as providing a local group for people to get together to talk
about their favorite books. Whether life changing or just a bit of fun, each
community has a belief system; the humble book club still sees tremendous value
in providing a fun, safe and free environment to share reading preferences and
recommendations. It might not change the world, but it is still a good thing and
something people can get behind.

The underlying often unwritten rule of community is that every contribution from
a community member must benefit the wider community. This is why it is fun to
write a patch that fixes a Free Software bug, contribute documentation, run a
free event or otherwise, but it is rare that anyone is willing to contribute as
a volunteer if their contribution only benefits a single person or company.

Of course, I am sure all of you cynical bastards are now going to try and find
an exception, but remember that this decision is typically deeply personal --
the community member decides how comfortable they are that their contribution
will benefit everyone. As an example, some would argue that any contribution to
Mono would only benefit Microsoft and the ubiquity of their .NET framework, but
hundreds of contributors participate in Mono because they do not see it this way
-- they see their contributions as a valuable and fun way of making it easy to
empower Free Software developers to write Free Software more easily.

If I was talking to the Jono of 1998 I would really emphasize the importance of
this belief. I had a hunch about it back then, but I have since seen countless
examples of belief truly inspiring people to participate. I have often talked
about the story of the kid from Africa who emailed me to tell me how he would
walk three hours to and from his nearest Internet cafe to contribute to Ubuntu.
He did this because he believed in our mission to bring Free Software to the
masses. The same can be said for the tremendous growth in Wikipedia, the
incredible coming together of the GNOME community around GNOME 3, the success of
OpenStreetMap and many other examples.

Belief though is not a PR stunt. It has to be real. While each of us has
different belief systems, some map their belief systems to software, some to
education, some to knowledge, some to transparency or whatever else, you ca not
concoct a belief system unless it serves a valid goal that a group are likely to
care about. Sure, it can be obscure, but it has to be real. With the success of
Open Source, we have seen some examples of some companies trying to use similar
language and approaches around belief, but applying it to self-serving needs. I
could invent a belief of ``let’s all work together to help Jono get rich'' and
concoct some nonsense of the benefits of this belief (e.g. if I am rich I can
focus on other work that would benefit other communities, my future kids would
get a wonderful education and upbringing and this will benefit the world), but
it would be rubbish.

As such, belief is a strong driver for collaboration and contribution, but it
must be met with respect and balance. While it can be a trigger for incredible
change, it can also be hugely destructive (e.g. some television preachers who
use religion as a means for you to give them money, or fake psychics who use
cold reading to latch onto your belief to desperately try and re-connect with a
lost loved one).

\section*{Your Role}

Community managers play an interesting role these days. In the past I have
talked about there being two types of community managers; those who go out and
give presentations and wave their hands around talking about a product or
service, and those who work with a community of volunteers to help them to have
a fun, productive and enjoyable collaborative experience. I am more interested
in the latter -- I feel that is what a real community manager does. The former
is a fine and respectable position to have, but it is more in the area of
advocacy and public relations, and requires a different set of skills. I have a
few tips here I think are interesting enough to share.

The first and probably most important lesson is having a willingness to accept
that you can and will be wrong sometimes. In my career so far I have got some
things right and some things wrong. While I believe I am generally on the right
path and most of my work is successful, there have been a few turkeys here and
there. These screw-ups, mishaps and mis-steps have never been out of
maliciousness or carelessness, they have instead typically been from me
overshooting the target of what I was trying to do.

This seems like a pretty obvious point, but it gets less obvious when you have a
fairly public role. By and large, community managers are often seen as
representatives of a given community. As an example, I know that I am personally
seen as one of the public faces of Ubuntu, and with that responsibility comes
the public pressure of how people perceive you.

For some community leaders, having the spotlight shone on them causes a
defensive mechanism to kick in; they cringe at the idea of making mistakes in
public, as if the chattering masses expect a perfect record. This is risky, and
what has been seen in the past is that we get public leaders who essentially
never accept that they have made a mistake due to this fear of public ridicule.
This is not only a fallacy (we all make mistakes), but it also does not set a
good example to the community of a leader who is honest and transparent in both
the things they do well and the things they do less well. It is important to
remember that we often gain respect in people because of their acceptance of
mistakes -- it shows a well-rounded and honest individual.

I remember when I first became a manager at Canonical and at the time Colin
Watson and Scott James Remnant, two original gangstas from the very beginning of
Canonical and Ubuntu, were also managers on the Ubuntu Engineering Team. We
would have our weekly calls with our manager, Matt Zimmerman, and on these calls
I would hear Colin and Scott openly accepting that they were not good at this,
or had made a mistake with that; they were stunningly humble and accepting of
their strengths and weaknesses. As a rookie manager I was a little more
tight-lipped, but it taught me that this kind of openness and honesty is not
only good as a manager but as a community leader and since then I feel no qualms
in publicly admitting to mistakes or apologizing if I screw up.

\section*{Listening}

In a similar way, while openness to mistakes is important, another lesson is the
importance of being a good listener and learning from our peers. In many cases
our communities look to community managers and leaders as people who should
always be providing guidance, direction and active navigation of the project and
its goals. This is definitely a responsibility, but in addition to the voicing
of this leadership, it is also important to be a passive listener, providing
guidance where appropriate and learning new lessons and insight.

Our community members are not just cold, hard, machines who perform work; they
are living, breathing, human beings with thoughts, opinions, feelings and ideas.
I have seen many examples, and I have accidentally done this before myself,
where someone is so used to providing guidance and direction that they sometimes
forget to just sit down and listen and learn from someone else’s experience.
Every industry is filled with thought leaders and scholars ... famous people
who are known for their wisdom, but in my experience some of the most
revolutionary life lessons that I have learned have come entirely from
non-famous, day-to-day, meat-and-potatoes community members. Being a great
listener is not just important to help us learn and be better at what we do, but
it is critical in gaining respect and having a great relationship with your
community.

\section*{On vs. Off Time}

While on the subject of how we engage with our community, I have another
take-away that I only truly processed in my own mind fairly recently. Like many
people, I have a number of different interests that fill my days. Outside of
being married and trying to be the best husband I can be, and my day job as the
Ubuntu Community Manager, I also have projects such as Severed Fifth, the
Community Leadership Summit, and some other things. As you would naturally
expect, my days are committed to my day job -- I do not spend time at work
working on these other projects. As such, as you would naturally expect, when my
work day ends I start working on these other projects. The lesson here is that
it is not always clear to your community where the lines are drawn.

Over the years I have developed a series of online facilities that I use for my
work and viewpoints. My Twitter, identi.ca, Facebook pages, my blog, and some
other resources are where I talk about what I do. The challenge is that if you
take into account these public resources, my public representation of the Ubuntu
project, and the wealth of timezones across the world, it does not take an
Einstein to confuse whether I am writing about something as a Jono thing or a
Canonical thing.

This has caused some confusion. As an example, despite my repeated
clarifications, OpenRespect is not and never has been a Canonical initiative. Of
course, some idiots choose to ignore my clarification of this, but I can see how
the confusion could arrive nonetheless. The same thing has happened for other
projects such as Severed Fifth, The Art of Community and the Community
Leadership Summit, of which none are, or ever have been, part of my work at
Canonical.

The reason why I consider this a lesson is that I have seen, and at one point
shared, the view that ``of course it is a spare time thing, I posted that at 8pm
at night'' and shrug of concerns of the lines blurring. When you have a job that
puts you in a reasonably public position, you can not have the luxury of just
assuming that; you have to instead assume that people are likely to blur the
lines and you have to work harder to clarify them.

\section*{Don’t Travel Too Much}

On the topic of working for a company that employs you to be a community leader,
you should always be aware of the risks as well as the benefits of travel. This
is something I learned fairly early on in my career at Canonical. I would see
the same faces over and over again at conferences, and it was clear that these
folks had clearly communicated the benefits of travel to their employer, as I
had done, but I also came to learn the risks.

I would travel and it would not only be tiring work and emotionally exhausting,
but I would also be away from my email more, on IRC less, unable to attend many
meetings, and have less time to work on my work commitments. As such, my role
would largely become that of getting out and visiting events, and while fun,
this did not serve my community as well as it should have done. As such, I
fairly dramatically cut my travel -- in fact, I went to the Linux Collab Summit
a few days ago, and outside of Ubuntu events that I needed to attend, I had not
made it to conference for nearly a year. Now I feel the pendulum has swung a
little too far in the other direction, so it is all about balance, but I also
feel I serve my community better when I am able to take the time to be at the
office and be online and accessible.

\section*{Planning}

For some folks, the role of a community leader or community manager is one that
is less about pre-disposed structure and instead more interrupt-driven. When I
started out, I used to think this too. While there is absolutely no doubt that
you do indeed need to be interrupt-driven and able to respond to things that are
going on, it is also essential to sufficiently plan your work for a given period
of time.

This planning should be done out in the open where possible and serves a few
functions:
\begin{itemize}
 \item Shares plans -- it helps the community to understand what you are working
on and often opens up the doors for the community to help you.
 \item Offers assurances -- it demonstrates that a community leader is doing something.
Your community can see your active work happening. This is particularly
important, as much of the work of a community leader often happens out of the
view of the wider community (e.g. having a one-on-one conversation with a
community member), and this lack of visibility can sometimes generate concerns
that little is happening in key areas, when instead a lot is going on behind the
scenes.
 \item Communicates progress up and down the ladder -- this is relevant if you
are working for a company. Having some solid planning processes in place
demonstrates your active work to your management, and it also re-assures your
team that they will always know what to work on and create great value for the
community.
\end{itemize}
Over the years I have put more and more importance in planning, while still
retaining enough time and flexibility to be interrupt-driven. When I started as
the Ubuntu Community Manager my planning was fairly personal and ad-hoc -- I
took the pulse of the community, and I applied my time and resources to tend to
those areas as I saw fit.

Today I break goals into a set of projects that each span an Ubuntu cycle,
gather input from stakeholders, put together a roadmap, track work in
blueprints, and assess progress using a variety of tools and processes such as
my burndown chart, regular meetings, and more. While the current approach
requires more planning, it helps significantly with the benefits covered in the
above bullet points.

\section*{Perception and Conflict}

One thing I often hear about in the world of community management and leadership
is the view that perception is everything. Typically when I hear this it is in
response to someone getting the wrong end of the stick about something, often in
a conflict period.

Of course, perception does indeed play an important part in our lives, but what
can fuel incorrect or misaligned perceptions is lack of information,
mis-information, and in some cases, heated tensions and tempers. This can be
some of the most complex work for a community leader, and I have come away with
a few lessons learned in this area too.

Communities are groups of people, and in every group there are often common
roles that people fill. There is usually someone who is seen as a rockstar and
hero, someone who is sympathetic to concerns and worries and a shoulder to cry
on, someone who is overtly outspoken, and often someone who is ... well ...
deliberately difficult. Heroes, sympathetic ears and outspoken folks are not
particularly challenging, but deliberately difficult people can be complex; if
someone is being overtly difficult to deal with, it can cause tensions to form
with other members and bring conflict to an otherwise happy community. We need
to nip those issues in the bud early.

Part of the challenge here is that people are people, groups are groups, and it
is not uncommon for a single person or a few people to become known and
complained about behind closed doors as difficult to work with. In addition to
this, most people do not want to get involved in any conflict, and as such the
person being complained about can sometimes never actually know that people see
them this way, as no-one wants to confront them about it. This results in one of
the most dangerous situations for a community members -- a reputation is spread,
without the knowledge of the person who it applies to, and because they never
know, they never have an opportunity to fix it. That is a pretty sucky position
to be in.

A common response to this conclusion is the view that ``they are so difficult to
deal with that trying to reason with them will fall on deaf ears anyway''. While
this certainly does happen from time to time, do not be so quick to assume this
will be the outcome; there have been a few times when I have had the
uncomfortable experience of feeling I need to share with someone the reputation
that they have developed, and in virtually all cases it has been a real surprise
to them, and they have almost all modified their behavior based on the feedback.

On a related note, while often not a common part of the daily routine of a
community leader, conflict will often raise its head here and there. I just
wanted to share two brief elements about conflict.

The first is understanding how conflict forms. To introduce this, let me tell
you a little story. Last week a friend of mine flew out to the Bay Area for a
conference. He arrived in the evening, so I picked him up from the airport and
we went to the pub to catch up. While there he started telling me how
disappointed he was with Obama and his administration. He cited examples of
health care reform, Wall Street reform, digital rights and more. His agitation
was not with the policies themselves, but with Obama not doing enough. My
perspective was a little different.

I am not a democrat or a republican; I make my decisions on each issue, and I do
not align myself with either party. Where I differ to my friend though is that I
am a little more sympathetic to Obama and his daily work. This is because I
believe that he, and anyone else in a public position, whether as
internationally recognized as the president, or as obscure and specific as a
community manager, realizes that the story read and understood by the public is
often only a fragment of the full story. There have been cases in the past where
something controversial has kicked off in the communities that I have been a
part of, and many of the commentators and onlookers have clearly not had a full
knowledge of the facts either because they have not picked up on the nuances and
details of the topic or some parts of the story have not been shared.

Now, I know what some of you are going to say -- some parts not shared?! Surely
we should be transparent? Of course we should, and we should always strive to be
open and honest, but there are some cases when it would be inappropriate to
share some parts of the story. This could be because of private conversations
with people who do not want their comments shared, and also just being classy in
your work and not throwing dirt around. As an example, I have always had a very
strong policy of not throwing cheap shots at competitors, no matter what
happens. In the past there has been some questionable behavior from some
competitors behind the scenes, but I am not going to go out and throw dirt
around as it would not serve a particularly useful purpose, but with that I have
to accept that some community critique will only have part of the picture and
not be aware of some of the behind the scenes shenanigans.

Finally, on the topic of conflict, I believe a real life lesson I have learned
has been the approach in which critique and successful outcomes should be
approached. Although blogging has had a hugely positive impact on how people can
articulate and share opinions and perspectives, there has been a dark side.
Blogging has also become a medium in which much overzealous opinion can
sometimes be expressed a little too quickly. Unfortunately, I have a rather
embarrassing example of someone who fell into this trap: yours truly.

First, a bit of background. There used to be a company called Lindows that made
a version of Linux that shared many visual and operational similarities to
Windows. Microsoft frowned at the name ``Lindows'', and a fight started to
change the name. Lindows initially resisted, but after mounting pressure,
changed their name to Linspire.

Now to the issue. Let me take the liberty to explain in the words of the article
itself:
\begin{quote}
 Recently a chap named Andrew Betts decided to take the non-free elements out of
Linspire and release the free parts as another Linspire-derived distribution
called Freespire. This act of re-releasing distributions or code is certainly
nothing new and is fully within the ethos of open source. In fact, many of the
distributions we use today were derived from existing tools.

Unfortunately, Linspire saw this as a problem and asked for the Freespire name
to be changed. Reading through the notice of the change, the language and flow
of the words screams marketing to me. I am certainly not insinuating that Betts
has been forced into writing the page, or that the Linspire marketing drones
have written it and appended his name, but it certainly doesn’t sound quite
right to me. I would have expected something along the lines of ``Freespire has
been changed to Squiggle to avoid confusion with the Linspire product'', but
this is not the case. Instead we are treated to choice marketing cuts such as
``To help alleviate any confusion, I contacted Linspire and they made an
extremely generous offer to us all''. Wow. What is this
one-chance-in-a-lifetime-not-sold-in-stores offer? Luckily, he continues, ``they
want everyone who has been following my project to experience ‘the real’
Linspire, FOR FREE!!!''. Now, pray tell, how do we get this ‘real‘ version of
the software ``FOR FREE!!!''?

``For a limited time, they are making available a coupon code called ‘FREESPIRE’
that will give you a free digital copy of Linspire! Please visit
http://linspire.com/freespire for details''. Oh \dots thanks.
\end{quote}

I gave Linspire a pretty full-throated kick in the wedding vegetables in my blog
entry. I told the story, objected to what I considered hypocrisy given their own
battle with similar-sounding trademarks, and vented. I wish Guitar Hero had
existed back then: it would have been a better use of my time.

I was wrong. My article was never going to achieve anything. Shortly after the
article was published, then-CEO Kevin Carmony emailed me. He was not a happy
bunny. His objection, and it was valid, was that I flew off the handle without
checking in with him first. My blog entry was my first reaction. The reality of
the story was far less dramatic, and Linspire were not the ogres that I painted
them to be. I apologized to Kevin and felt like an idiot.

Many conflict scenarios are resolved in private discussions where people can be
open and focus on solutions without the noise. Over the years I have seen many
examples of a furious public blogging war going on while behind the scenes there
is a calm exchange of opinions and the focus on solutions.

\section*{Wrapping Up}

When I started writing this it was much shorter, but I just kept adding one more
thing, and then one more thing and so on. It is already long enough that I can
probably count the number of people reading this bit on one hand, so I am going
to hang it up here. I could go on forever with little tidbits and experiences
that I have been fortunate enough to be involved in and expand my horizons, but
then I would end up writing The Art of Community II: This Time It's Personal.

Life is a constant on-going experience, and I hope your investment in reading
this has added to it a little.
