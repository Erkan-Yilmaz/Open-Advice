\section*{Foreword -- Georg Greve}
\todo{include license in text}
\todo{book still needs a good end - some closing words?}

This is a book about community and technology. It is a book that
represents a collective effort, much like the technology we build
together. And if this is in fact your first encounter with our
community, you may find it strange to think of a community as the
driving force behind technology. Isn't technology built by large
corporations? Actually, for us it is almost the other way around.

The authors in this book are all members of what you could label the
software freedom community. A group of people sharing the fundamental
experience that software is more empowering, more useful, more
flexible, more controllable, more just, more encompassing, more
sustainable, more efficient, more secure and ultimately just better
when it comes with four fundamental freedoms: to use, study, share and
improve the software.

While there is now an increasing number of communities that have left
behind the requirement for geographical proximity by means of virtual
communication, it was this community that pioneered that new age. 

In fact, the Internet and the Free Software Community\footnote{For me,
  Open Source is one aspect of that community which articulated itself
  in 1998, so quite some time after the Internet came about. But
  please feel free to replace Free Software by Open Source in your
  head if that is your preferred terminology.}  were co-dependent
developments. As the Internet grew, our community could grow with it,
but without the values and technology of our community, I have no doubts
that the Internet would not have become the
all-encompassing network that we now see enabling people and groups
around the world.

Until today, our software runs most of the Internet, and you will know
at least some of it, such as Mozilla Firefox,
OpenOffice.org, Linux, and perhaps even GNOME or KDE. But our
technology may also be hidden inside your TV, your wireless
router, your ATM, even your radio, security system or battleships. It
is literally everywhere. 

It was essential in enabling quite some of the large corporations
that you know, such as Google, Facebook, Twitter and others. None of
these could have achieved so much in such a short time if it were not
for the power of software freedom that allowed them to stand on the
shoulders of those who came before.

But there are many smaller companies that live from, with, and for
Free Software, including my own, Kolab Systems. Active partaking in
the community in good faith and standing has become a critical success
factor for all of us. And this is true even for the large ones, as
Oracle has involuntarily demonstrated during and after its takeover of
Sun Microsystems.

But it is important to understand that our community is \textbf{not}
anti commercial. We enjoy our work, and many of us have made it their
profession for their livelihood and mortgage. So when we say
community, we mean students, entrepreneurs, developers, artists,
documentation writers, teachers, tinkerers, businessmen, sales people,
volunteers and users.

Yes, users. Even if you did not realize it or never signed up for no
community, you in fact are already \emph{almost} part of ours. The
question is whether you'll choose to participate actively.

And this is what sets us apart from the monoculture behemoths, the
gated communities, the corporate owned walled gardens of companies
like Apple, Microsoft and others. Our doors are open. So is our
advice. And your potential. There is no limit as to what you can
become -- it purely depends on your personal choice as it has
depended for each of us.

So if you are not yet part of our community, or simply curious, this
book provides a good starting point. And if you are already an active
participant, this book might provide you with insights into a few
facets and perspectives that are new to you.

Because this book contains important grains of that implicit
knowledge which we usually build and transfer inside our
sub-communities that work on different technologies. This knowledge
typically trickles down from experienced contributors to less
experienced ones, which is why it seems very obvious and
natural to those socialized in our community.

This is the same knowledge and culture of how to shape collaboration
that allows us to build outstanding technology in small, distributed
teams across language, country and cultural barriers around the world,
outperforming much larger development teams in some of the world's
largest corporations.

All the people writing in this book are such experienced contributors
in one, sometimes several areas. They have grown to become teachers
and mentors. Over the course of the past 15 years or so I had the
pleasure of getting to know most of them, working with many, and the
privilege to call some of them friends. Because as Kevin Ottens
rightly said during the Desktop Summit 2011 in Berlin: ``Community
building is family and friendship building.''

So it is in fact with a profound sense of gratitude that I can say
there is no other community I would rather be part of, and I look
forward to hopefully seeing you at one or the other next conference.
\newline
\begin{flushright}--- Georg Greve\end{flushright}
\begin{flushright}Zürich, Switzerland; 20. August 2011\end{flushright}

\textit{Georg Greve initiated the Free Software Foundation Europe in
2000 and was its founding president until 2009. During this time he
was responsible for building up and designing many of FSFE's activities
such as the Fellowship, the policy or legal work, and has worked
intensively with many communities. Today he continues this work as
shareholder and CEO of Kolab Systems AG, a fully Free Software company.
For his accomplishments in Free Software and Open Standards Georg
Greve was awarded the Federal Cross of Merit on ribbon by the Federal
Republic of Germany on 18 December 2009.}

\newpage
